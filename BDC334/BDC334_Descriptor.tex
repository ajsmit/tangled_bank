\documentclass[a4paper,9pt]{extarticle}

\usepackage[margin=14mm]{geometry}
\usepackage{fontspec}
\usepackage{longtable,booktabs,array,makecell,enumitem,microtype}

\setmainfont{TeX Gyre Termes}

\setlength{\parindent}{0pt}
\setlist[itemize]{leftmargin=1.1em, topsep=1pt, itemsep=1pt, parsep=0pt}
\setlength{\LTpre}{0pt}
\setlength{\LTpost}{6pt}
\renewcommand{\arraystretch}{1.02}

\renewcommand\theadfont{\normalsize\bfseries}
\setlength{\tabcolsep}{4pt}

\begin{document}

\begin{longtable}{@{}>{\raggedright\arraybackslash}p{0.30\linewidth} >{\raggedright\arraybackslash}p{0.62\linewidth}@{}}
\toprule
\multicolumn{2}{>{\raggedright\arraybackslash}p{\dimexpr\linewidth-2\tabcolsep\relax}}{\Large\textbf{BDC334 — Biogeography and Global Ecology 334}}\\
\midrule
\endfirsthead
\toprule
\multicolumn{2}{>{\raggedright\arraybackslash}p{\dimexpr\linewidth-2\tabcolsep\relax}}{\Large\textbf{BDC334 — Biogeography and Global Ecology 334} (continued)}\\
\midrule
\endhead
\midrule
\multicolumn{2}{r}{\emph{Table continues on next page}}
\\\bottomrule
\endfoot
\bottomrule
\endlastfoot

\textbf{Faculty} & Natural Sciences\\
\textbf{Home Department} & Biodiversity and Conservation Biology\\
\textbf{Module Topic} & Biogeography and Global Ecology\\
\textbf{Generic Module Name} & Biogeography and Global Ecology 334\\
\textbf{Alpha-numeric Code} & BDC334\\
\textbf{NQF Level} & 7\\
\textbf{NQF Credit Value} & 30\\
\textbf{Duration} & Semester\\
\textbf{Proposed semester offered} & Second Semester\\
\textbf{Programmes} & BSc (Biodiversity and Conservation Biology) (3217, 3015)\\
\textbf{Year level} & 3\\[1pt]

\textbf{Main Outcomes} &
\begin{itemize}
  \item Discuss the past, present and projected future patterns of global biogeography.
  \item Examine the distribution of past floras, faunas and climate with respect to plate tectonics and compare them with current distributions.
  \item Explain the role that the major environmental drivers play in driving biogeographical patterns.
  \item Understand the physical basis underpinning the components of global change.
  \item Recognise the central importance that humans play in bringing about global change.
  \item Understand the ecological, physiological and behavioural basis for biogeographical change.
  \item Contrast the fundamental differences between ecological biogeography and historical biogeography.
  \item Consider the biogeography of key extant plant and animal lineages.
  \item Apply appropriate concepts to collect, analyse and interpret multivariate environmental and ecological data.
  \item Present their position on the above in discussion or in written format.
\end{itemize}
\\[2pt]

\textbf{Main Content} &
\begin{itemize}
  \item Global biogeography: key principles and concepts.
  \item Continental drift and glaciation.
  \item Theories of biogeography and biogeographic reconstruction.
  \item Phylogeography.
  \item Latitudinal gradients in diversity.
  \item Interactions of body and population size on diversity and distribution.
  \item Island biogeography theory and its applications for conservation.
  \item Earth as a system.
  \item The physical nature of environmental drivers of biogeography.
  \item Global change: the distinction between natural variability and anthropogenically-driven change.
  \item Overview of the biological responses to global change.
  \item Basic data collection and analytical methods in biogeography.
\end{itemize}
\\[2pt]

\textbf{Pre-requisite Modules} & BDC211 and BDC221 and BDC223\\
\textbf{Co-requisite Modules} & None\\
\textbf{Prohibited Module Combination} & None\\[2pt]

\textbf{Breakdown of Learning Time} &
\begin{tabular}{@{}>{\raggedright\arraybackslash}p{0.44\linewidth} >{\raggedright\arraybackslash}p{0.12\linewidth} >{\raggedright\arraybackslash}p{0.34\linewidth}@{}}
\toprule
\textbf{Component} & \textbf{Hours} & \parbox{\linewidth}{\raggedright\textbf{Time-table requirement / other modes}}\\
\midrule
Contact with lecturer / tutor & 42 & Lectures p.w.: 3\\
Assignments \& tasks & 64 & \\
Practicals & 84 & Practicals p.w.: 2$\times$3\\
Assessments & 10 & Tutorials p.w.: 1\\
Self-study & 100 & \\
Other & 0 & \\
\midrule
\textbf{Total Learning Time} & \textbf{300} & \\
\bottomrule
\end{tabular}
\\[6pt]

\textbf{Method of Student Assessment} &
\parbox{\linewidth}{\raggedright Continuous Assessment (CA): 60\%\\Final Assessment (FA): 40\%}\\
\textbf{Assessment Module Type} & Continuous and Final Assessment (CFA)\\

\end{longtable}

\end{document}
