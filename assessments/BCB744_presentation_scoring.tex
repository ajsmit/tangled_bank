\documentclass[a4paper,10pt]{article}
\usepackage{graphicx}
\usepackage{array}
\usepackage{booktabs}
\usepackage{longtable}
\usepackage{geometry}
\usepackage[raggedright]{ragged2e}
\geometry{left=1in, right=1in, top=1in, bottom=1in}

% --- Custom Column Type (using ragged2e) ---

\newcolumntype{R}[1]{>{\RaggedRight}p{#1}}

\usepackage{fontspec}         % Allows font specification
\setmainfont[
  Renderer=Basic, % Recommended for plain text
  UprightFont = MinionPro-Regular,
  ItalicFont = MinionPro-It,
  ItalicFeatures = { SmallCapsFont = MinionPro-It },
  SlantedFont = MinionPro-Regular,
  SlantedFeatures= { FakeSlant=0.2 },
  BoldFont = MinionPro-Bold,
  BoldFeatures = { SmallCapsFont = MinionPro-Bold },
  BoldItalicFont = MinionPro-BoldIt,
  BoldItalicFeatures = { SmallCapsFont = MinionPro-BoldIt },
  BoldSlantedFont= MinionPro-Bold,
  BoldSlantedFeatures= { FakeSlant=0.2, SmallCapsFont = MinionPro-Bold },
  SmallCapsFont = MinionPro-Regular,
  SmallCapsFeatures={ RawFeature=+smcp },
  Ligatures=TeX,
  Numbers={OldStyle, Proportional}
]{MinionPro-Regular}

% --- Other Typography Settings ---

\frenchspacing                % Single space after periods
\tolerance=400                % Default is 200; higher values allow more relaxed line-breaking.
\emergencystretch=3em         % Adds additional space to help line-breaking.
\hyphenpenalty=20             % Default is 50; lower values encourage hyphenation.

% --- Microtype Settings (adjust only if needed) ---

\usepackage{microtype}        % Improves typography
\microtypesetup{
   tracking = true,
   protrusion=true,
   expansion=true,
   factor = 1100,
   stretch = 15,
   shrink = 15
}

\title{\textbf{BCB744 Presentation Assessment Form}}
\author{}
\date{}

\begin{document}
\maketitle

\section*{Student Name: \underline{\hspace{5cm}}}

\section*{Assessment Criteria}
Each presentation will be evaluated based on the following criteria, with a total possible score of 100\%. Marks are allocated to reflect both the depth of thought and the effectiveness of communication.

{\small
\begin{longtable}{R{3.5cm} R{8.5cm} R{3cm}}
    \toprule
    \textbf{Criterion} & \textbf{Description} & \textbf{Marks Allocated} \\
    \midrule
    \textbf{Depth of Understanding} & Demonstrates a clear, well-developed grasp of the topic, moving beyond surface-level descriptions. Provides logical, well-reasoned analysis, not just factual summaries. Use well-selected evidence, if necessary. & 30 \% \\
    \textbf{Originality \& Critical Thinking} & Shows independent thought rather than reproducing existing information. Challenges assumptions, explores alternative viewpoints, and constructs well-supported, persuasive arguments. & 25 \% \\
    \textbf{Communication \& Delivery} & Presents with clarity, confidence, and engagement. Avoids reading from slides, maintains good pacing, and articulates ideas logically. Uses appropriate tone, vocabulary, and emphasis to enhance comprehension. & 20 \% \\
    \textbf{Slide Quality \& Structure} & Slides are well-organised, visually clear, and enhance understanding. They follow a logical flow, avoiding unnecessary clutter or excessive text. Consistent aesthetic choices (font, spacing, color contrast) enhance readability and engagement. & 15 \% \\
    \textbf{Time Management \& Adherence to Guidelines} & Presentation stays within the 5 to 7-minute limit. Slides are submitted on time, in the correct format, and no notes are used. & 10 \% \\
    \bottomrule
\end{longtable}}

\section*{Scoring Guide}
Each criterion is scored according to the following rubric:

{\small
\begin{longtable}{R{3.5cm} R{11cm}}
    \toprule
    \textbf{Score Range} & \textbf{Description} \\
    \midrule
    \textbf{Exceptional} (90-100\%) & Demonstrates deep understanding, original insights, polished delivery, and excellent slide design. Completely adheres to guidelines. \\
    \textbf{Good} (75-89\%) & Strong understanding with some original insights, clear communication, and mostly well-structured slides. Minor guideline lapses. \\
    \textbf{Adequate} (60-74\%) & Sufficient grasp of topic but lacks depth or originality. Communication is adequate but not highly engaging. Some issues with slides. \\
    \textbf{Weak} (40-59\%) & Limited understanding with minimal critical thinking. Delivery is unclear or disorganised, and slides are poorly structured. \\
    \textbf{Poor} (0-39\%) & Misunderstands key aspects, lacks depth, and relies on rote memorisation. Delivery is weak, slides are confusing, and guidelines are ignored. \\
    \bottomrule
\end{longtable}}

\section*{Final Score: \underline{\hspace{2cm}} / 100\%}

This structured assessment ensures fairness and consistency, emphasising originality, depth of analysis, and effective communication.

\end{document}
