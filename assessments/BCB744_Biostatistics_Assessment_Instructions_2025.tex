% Options for packages loaded elsewhere
% Options for packages loaded elsewhere
\PassOptionsToPackage{unicode}{hyperref}
\PassOptionsToPackage{hyphens}{url}
\PassOptionsToPackage{dvipsnames,svgnames,x11names}{xcolor}
%
\documentclass[
  10pt,
]{article}
\usepackage{xcolor}
\usepackage{amsmath,amssymb}
\setcounter{secnumdepth}{-\maxdimen} % remove section numbering
\usepackage{iftex}
\ifPDFTeX
  \usepackage[T1]{fontenc}
  \usepackage[utf8]{inputenc}
  \usepackage{textcomp} % provide euro and other symbols
\else % if luatex or xetex
  \usepackage{unicode-math} % this also loads fontspec
  \defaultfontfeatures{Scale=MatchLowercase}
  \defaultfontfeatures[\rmfamily]{Ligatures=TeX,Scale=1}
\fi
\usepackage{lmodern}
\ifPDFTeX\else
  % xetex/luatex font selection
\fi
% Use upquote if available, for straight quotes in verbatim environments
\IfFileExists{upquote.sty}{\usepackage{upquote}}{}
\IfFileExists{microtype.sty}{% use microtype if available
  \usepackage[]{microtype}
  \UseMicrotypeSet[protrusion]{basicmath} % disable protrusion for tt fonts
}{}
\makeatletter
\@ifundefined{KOMAClassName}{% if non-KOMA class
  \IfFileExists{parskip.sty}{%
    \usepackage{parskip}
  }{% else
    \setlength{\parindent}{0pt}
    \setlength{\parskip}{6pt plus 2pt minus 1pt}}
}{% if KOMA class
  \KOMAoptions{parskip=half}}
\makeatother
% Make \paragraph and \subparagraph free-standing
\makeatletter
\ifx\paragraph\undefined\else
  \let\oldparagraph\paragraph
  \renewcommand{\paragraph}{
    \@ifstar
      \xxxParagraphStar
      \xxxParagraphNoStar
  }
  \newcommand{\xxxParagraphStar}[1]{\oldparagraph*{#1}\mbox{}}
  \newcommand{\xxxParagraphNoStar}[1]{\oldparagraph{#1}\mbox{}}
\fi
\ifx\subparagraph\undefined\else
  \let\oldsubparagraph\subparagraph
  \renewcommand{\subparagraph}{
    \@ifstar
      \xxxSubParagraphStar
      \xxxSubParagraphNoStar
  }
  \newcommand{\xxxSubParagraphStar}[1]{\oldsubparagraph*{#1}\mbox{}}
  \newcommand{\xxxSubParagraphNoStar}[1]{\oldsubparagraph{#1}\mbox{}}
\fi
\makeatother


\usepackage{longtable,booktabs,array}
\usepackage{calc} % for calculating minipage widths
% Correct order of tables after \paragraph or \subparagraph
\usepackage{etoolbox}
\makeatletter
\patchcmd\longtable{\par}{\if@noskipsec\mbox{}\fi\par}{}{}
\makeatother
% Allow footnotes in longtable head/foot
\IfFileExists{footnotehyper.sty}{\usepackage{footnotehyper}}{\usepackage{footnote}}
\makesavenoteenv{longtable}
\usepackage{graphicx}
\makeatletter
\newsavebox\pandoc@box
\newcommand*\pandocbounded[1]{% scales image to fit in text height/width
  \sbox\pandoc@box{#1}%
  \Gscale@div\@tempa{\textheight}{\dimexpr\ht\pandoc@box+\dp\pandoc@box\relax}%
  \Gscale@div\@tempb{\linewidth}{\wd\pandoc@box}%
  \ifdim\@tempb\p@<\@tempa\p@\let\@tempa\@tempb\fi% select the smaller of both
  \ifdim\@tempa\p@<\p@\scalebox{\@tempa}{\usebox\pandoc@box}%
  \else\usebox{\pandoc@box}%
  \fi%
}
% Set default figure placement to htbp
\def\fps@figure{htbp}
\makeatother





\setlength{\emergencystretch}{3em} % prevent overfull lines

\providecommand{\tightlist}{%
  \setlength{\itemsep}{0pt}\setlength{\parskip}{0pt}}



 


% --- Your modified preamble.tex begins here ---
\usepackage[a4paper]{geometry}
\frenchspacing
\tolerance=400
\emergencystretch=3em
\hyphenpenalty=20
\usepackage{microtype}
\microtypesetup{
  tracking   = true,
  protrusion = true,
  expansion  = true,
  factor     = 1100,
  stretch    = 15,
  shrink     = 15,
}
\let\oldtexttt\texttt
\renewcommand{\texttt}[1]{\oldtexttt{\small #1}}
\usepackage{etoolbox}
\AtBeginEnvironment{Highlighting}{\footnotesize}
\AtBeginEnvironment{verbatim}{\footnotesize}
\AtBeginEnvironment{Shaded}{\footnotesize}
\makeatletter
\@ifpackageloaded{caption}{}{\usepackage{caption}}
\AtBeginDocument{%
\ifdefined\contentsname
  \renewcommand*\contentsname{Table of contents}
\else
  \newcommand\contentsname{Table of contents}
\fi
\ifdefined\listfigurename
  \renewcommand*\listfigurename{List of Figures}
\else
  \newcommand\listfigurename{List of Figures}
\fi
\ifdefined\listtablename
  \renewcommand*\listtablename{List of Tables}
\else
  \newcommand\listtablename{List of Tables}
\fi
\ifdefined\figurename
  \renewcommand*\figurename{Figure}
\else
  \newcommand\figurename{Figure}
\fi
\ifdefined\tablename
  \renewcommand*\tablename{Table}
\else
  \newcommand\tablename{Table}
\fi
}
\@ifpackageloaded{float}{}{\usepackage{float}}
\floatstyle{ruled}
\@ifundefined{c@chapter}{\newfloat{codelisting}{h}{lop}}{\newfloat{codelisting}{h}{lop}[chapter]}
\floatname{codelisting}{Listing}
\newcommand*\listoflistings{\listof{codelisting}{List of Listings}}
\makeatother
\makeatletter
\makeatother
\makeatletter
\@ifpackageloaded{caption}{}{\usepackage{caption}}
\@ifpackageloaded{subcaption}{}{\usepackage{subcaption}}
\makeatother
\makeatletter
\@ifpackageloaded{sidenotes}{}{\usepackage{sidenotes}}
\@ifpackageloaded{marginnote}{}{\usepackage{marginnote}}
\makeatother
\usepackage{bookmark}
\IfFileExists{xurl.sty}{\usepackage{xurl}}{} % add URL line breaks if available
\urlstyle{same}
\hypersetup{
  pdftitle={BCB744 Practical Exam Assessment Instructions (2025)},
  pdfauthor={Smit, A. J.},
  colorlinks=true,
  linkcolor={blue},
  filecolor={blue},
  citecolor={blue},
  urlcolor={blue},
  pdfcreator={LaTeX via pandoc}}


\title{BCB744 Practical Exam Assessment Instructions (2025)}
\author{Smit, A. J.}
\date{2025-05-31}
\begin{document}
\maketitle


Assessment Instructions for BCB744 Practical Exam (2025)

\subsection{INPUT}\label{input}

\begin{itemize}
\tightlist
\item
  The rubric is defined in BCB744\_Prac\_Exam\_Rubric\_2025.pdf
  (attached once only at the start).
\item
  The worked out answers which will guide the assessment in
  BCB744\_Biostats\_Proac\_Exam\_2025.pdf (attached once at the start)
\item
  Each student's response will be in a .html, .docx, or .pdf output
  file.
\item
  Assessment criteria apply per task and question, with overall
  weightings per task provided.
\end{itemize}

\subsection{STEP-BY-STEP ASSESSMENT
PROCEDURE}\label{step-by-step-assessment-procedure}

\subsubsection{STEP 1: Parse and identify the student
file}\label{step-1-parse-and-identify-the-student-file}

\begin{enumerate}
\def\labelenumi{\arabic{enumi}.}
\tightlist
\item
  Read the student answer file.
\item
  Identify and extract answers corresponding to:
\end{enumerate}

\begin{itemize}
\tightlist
\item
  Task 1 (with subcomponents 1.1 and 1.2)
\item
  Task 2.1 (1 and 2), 2.2 (1, 2, 3), and 2.3
\item
  Task 3 (1--4)
\item
  Task 4.1 and 4.2
\item
  Task 5.1 through 5.5
\item
  Task 6 (Write-up)
\end{itemize}

\subsubsection{STEP 2: Evaluate each component using the
rubric}\label{step-2-evaluate-each-component-using-the-rubric}

For each sub-question or component:

\begin{enumerate}
\def\labelenumi{\arabic{enumi}.}
\tightlist
\item
  Apply the rubric section relevant to that task:
\end{enumerate}

\begin{itemize}
\tightlist
\item
  Use the four assessment dimensions:

  \begin{itemize}
  \tightlist
  \item
    Technical Accuracy (50\%)
  \item
    Depth of Analysis (20\%)
  \item
    Clarity and Communication (20\%)
  \item
    Critical Thinking (10\%)
  \end{itemize}
\item
  Each is scored on a 0--100 scale for that component.
\end{itemize}

\begin{enumerate}
\def\labelenumi{\arabic{enumi}.}
\setcounter{enumi}{1}
\tightlist
\item
  Multiply each score by the weighting for that component as defined in
  the rubric:
\end{enumerate}

\begin{itemize}
\tightlist
\item
  E.g., Task 1.1 is 50\% of Task 1 (worth 10\%), so max contribution is
  5 points.
\item
  Task 5.3 is one of five sub-tasks in Task 5 (30\% total), so it's
  \textasciitilde6\%.
\end{itemize}

\begin{enumerate}
\def\labelenumi{\arabic{enumi}.}
\setcounter{enumi}{2}
\tightlist
\item
  Tally sub-task scores to compute the task total (e.g., Task 3 might
  yield 17.4/20).
\item
  Round task scores to one decimal place.
\end{enumerate}

\subsubsection{STEP 3: Write feedback and save to
.txt}\label{step-3-write-feedback-and-save-to-.txt}

For each student, generate a .txt file named identically to their input
file (but with .txt extension):

A. Feedback Report Structure

\begin{enumerate}
\def\labelenumi{\arabic{enumi}.}
\tightlist
\item
  Narrative feedback for each task (Tasks 1--6)
\end{enumerate}

\begin{itemize}
\tightlist
\item
  One paragraph per task.
\item
  Highlight:

  \begin{itemize}
  \tightlist
  \item
    Strengths (e.g., well-structured code, clear visualisations)
  \item
    Weaknesses (e.g., incorrect model use, insufficient explanation)
  \item
    Areas for improvement (e.g., mention VIF or DW test next time)
  \item
    Must be constructive and written for student learning.
  \end{itemize}
\end{itemize}

\begin{enumerate}
\def\labelenumi{\arabic{enumi}.}
\setcounter{enumi}{1}
\tightlist
\item
  Marks per component
\end{enumerate}

\begin{itemize}
\tightlist
\item
  Use format: Task 1.1: 43/50 or Task 2.2 (2): 12/20
\item
  One line per sub-question (lowest possible granularity)
\end{itemize}

\begin{enumerate}
\def\labelenumi{\arabic{enumi}.}
\setcounter{enumi}{2}
\tightlist
\item
  Task total
\end{enumerate}

\begin{itemize}
\tightlist
\item
  Use format: Task 1: 8.6/10
\end{itemize}

\begin{enumerate}
\def\labelenumi{\arabic{enumi}.}
\setcounter{enumi}{3}
\tightlist
\item
  Final total
\end{enumerate}

\begin{itemize}
\tightlist
\item
  Use format: Total mark: 84.5/100
\end{itemize}

\subsubsection{STEP 4: Generate .csv with
marks}\label{step-4-generate-.csv-with-marks}

For the same student, create a .csv file (named identically but with
.csv extension) with the following structure:

Task Mark Task 1 8.6 Task 2 9.2 Task 3 17.4 Task 4 9.0 Task 5 27.0 Task
6 9.3 Total 80.5

\subsection{ADDITIONAL GUIDELINE FOR
CONSISTENCY}\label{additional-guideline-for-consistency}

\begin{itemize}
\tightlist
\item
  Use the same rubric for all students.
\item
  Apply point deductions proportionally across the four dimensions of
  the rubric.
\item
  Do not penalise for choices beyond the scope of the taught material
  (e.g., not using mixed models).
\item
  Award partial marks for attempts that demonstrate correct reasoning,
  even if syntax is flawed.
\item
  Always refer to the original ``Notes to Assessor'' where included for
  guidance on expected answers.
\end{itemize}

\subsection{SUMMARY}\label{summary}

Output Type Content:

\begin{itemize}
\tightlist
\item
  .txt Narrative feedback, component marks, task marks, total mark
\item
  .csv Tabular summary of marks per task + total
\end{itemize}




\end{document}
