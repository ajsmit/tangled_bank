% Options for packages loaded elsewhere
% Options for packages loaded elsewhere
\PassOptionsToPackage{unicode}{hyperref}
\PassOptionsToPackage{hyphens}{url}
\PassOptionsToPackage{dvipsnames,svgnames,x11names}{xcolor}
%
\documentclass[
  10pt,
]{article}
\usepackage{xcolor}
\usepackage{amsmath,amssymb}
\setcounter{secnumdepth}{-\maxdimen} % remove section numbering
\usepackage{iftex}
\ifPDFTeX
  \usepackage[T1]{fontenc}
  \usepackage[utf8]{inputenc}
  \usepackage{textcomp} % provide euro and other symbols
\else % if luatex or xetex
  \usepackage{unicode-math} % this also loads fontspec
  \defaultfontfeatures{Scale=MatchLowercase}
  \defaultfontfeatures[\rmfamily]{Ligatures=TeX,Scale=1}
\fi
\usepackage{lmodern}
\ifPDFTeX\else
  % xetex/luatex font selection
\fi
% Use upquote if available, for straight quotes in verbatim environments
\IfFileExists{upquote.sty}{\usepackage{upquote}}{}
\IfFileExists{microtype.sty}{% use microtype if available
  \usepackage[]{microtype}
  \UseMicrotypeSet[protrusion]{basicmath} % disable protrusion for tt fonts
}{}
\makeatletter
\@ifundefined{KOMAClassName}{% if non-KOMA class
  \IfFileExists{parskip.sty}{%
    \usepackage{parskip}
  }{% else
    \setlength{\parindent}{0pt}
    \setlength{\parskip}{6pt plus 2pt minus 1pt}}
}{% if KOMA class
  \KOMAoptions{parskip=half}}
\makeatother
% Make \paragraph and \subparagraph free-standing
\makeatletter
\ifx\paragraph\undefined\else
  \let\oldparagraph\paragraph
  \renewcommand{\paragraph}{
    \@ifstar
      \xxxParagraphStar
      \xxxParagraphNoStar
  }
  \newcommand{\xxxParagraphStar}[1]{\oldparagraph*{#1}\mbox{}}
  \newcommand{\xxxParagraphNoStar}[1]{\oldparagraph{#1}\mbox{}}
\fi
\ifx\subparagraph\undefined\else
  \let\oldsubparagraph\subparagraph
  \renewcommand{\subparagraph}{
    \@ifstar
      \xxxSubParagraphStar
      \xxxSubParagraphNoStar
  }
  \newcommand{\xxxSubParagraphStar}[1]{\oldsubparagraph*{#1}\mbox{}}
  \newcommand{\xxxSubParagraphNoStar}[1]{\oldsubparagraph{#1}\mbox{}}
\fi
\makeatother


\usepackage{longtable,booktabs,array}
\usepackage{calc} % for calculating minipage widths
% Correct order of tables after \paragraph or \subparagraph
\usepackage{etoolbox}
\makeatletter
\patchcmd\longtable{\par}{\if@noskipsec\mbox{}\fi\par}{}{}
\makeatother
% Allow footnotes in longtable head/foot
\IfFileExists{footnotehyper.sty}{\usepackage{footnotehyper}}{\usepackage{footnote}}
\makesavenoteenv{longtable}
\usepackage{graphicx}
\makeatletter
\newsavebox\pandoc@box
\newcommand*\pandocbounded[1]{% scales image to fit in text height/width
  \sbox\pandoc@box{#1}%
  \Gscale@div\@tempa{\textheight}{\dimexpr\ht\pandoc@box+\dp\pandoc@box\relax}%
  \Gscale@div\@tempb{\linewidth}{\wd\pandoc@box}%
  \ifdim\@tempb\p@<\@tempa\p@\let\@tempa\@tempb\fi% select the smaller of both
  \ifdim\@tempa\p@<\p@\scalebox{\@tempa}{\usebox\pandoc@box}%
  \else\usebox{\pandoc@box}%
  \fi%
}
% Set default figure placement to htbp
\def\fps@figure{htbp}
\makeatother





\setlength{\emergencystretch}{3em} % prevent overfull lines

\providecommand{\tightlist}{%
  \setlength{\itemsep}{0pt}\setlength{\parskip}{0pt}}



 


% --- Your modified preamble.tex begins here ---
\usepackage[a4paper]{geometry}
\frenchspacing
\tolerance=400
\emergencystretch=3em
\hyphenpenalty=20

\usepackage{microtype}
\microtypesetup{
  tracking   = true,
  protrusion = true,
  expansion  = true,
  factor     = 1100,
  stretch    = 15,
  shrink     = 15,
}

\let\oldtexttt\texttt
\renewcommand{\texttt}[1]{\oldtexttt{\small #1}}
\usepackage{etoolbox}
\AtBeginEnvironment{Highlighting}{\footnotesize}
\AtBeginEnvironment{verbatim}{\footnotesize}
\AtBeginEnvironment{Shaded}{\footnotesize}
\makeatletter
\@ifpackageloaded{caption}{}{\usepackage{caption}}
\AtBeginDocument{%
\ifdefined\contentsname
  \renewcommand*\contentsname{Table of contents}
\else
  \newcommand\contentsname{Table of contents}
\fi
\ifdefined\listfigurename
  \renewcommand*\listfigurename{List of Figures}
\else
  \newcommand\listfigurename{List of Figures}
\fi
\ifdefined\listtablename
  \renewcommand*\listtablename{List of Tables}
\else
  \newcommand\listtablename{List of Tables}
\fi
\ifdefined\figurename
  \renewcommand*\figurename{Figure}
\else
  \newcommand\figurename{Figure}
\fi
\ifdefined\tablename
  \renewcommand*\tablename{Table}
\else
  \newcommand\tablename{Table}
\fi
}
\@ifpackageloaded{float}{}{\usepackage{float}}
\floatstyle{ruled}
\@ifundefined{c@chapter}{\newfloat{codelisting}{h}{lop}}{\newfloat{codelisting}{h}{lop}[chapter]}
\floatname{codelisting}{Listing}
\newcommand*\listoflistings{\listof{codelisting}{List of Listings}}
\makeatother
\makeatletter
\makeatother
\makeatletter
\@ifpackageloaded{caption}{}{\usepackage{caption}}
\@ifpackageloaded{subcaption}{}{\usepackage{subcaption}}
\makeatother
\makeatletter
\@ifpackageloaded{sidenotes}{}{\usepackage{sidenotes}}
\@ifpackageloaded{marginnote}{}{\usepackage{marginnote}}
\makeatother
\usepackage{bookmark}
\IfFileExists{xurl.sty}{\usepackage{xurl}}{} % add URL line breaks if available
\urlstyle{same}
\hypersetup{
  pdftitle={BCB744 Biostatistics Exam Rubric (2025)},
  pdfauthor={Smit, A. J.},
  colorlinks=true,
  linkcolor={blue},
  filecolor={blue},
  citecolor={blue},
  urlcolor={blue},
  pdfcreator={LaTeX via pandoc}}


\title{BCB744 Biostatistics Exam Rubric (2025)}
\author{Smit, A. J.}
\date{2025-05-31}
\begin{document}
\maketitle


General Structure of the Rubric

Each Task is evaluated under the following axes:

\begin{enumerate}
\def\labelenumi{\arabic{enumi}.}
\tightlist
\item
  Technical Accuracy (50\%)
\item
  Depth of Analysis (20\%)
\item
  Clarity and Communication (20\%)
\item
  Critical Thinking (10\%)
\end{enumerate}

Each subcomponent is marked on a 0--100 scale, then scaled to its
proportion of the task weight. For example, Task 5 is worth 30\% of the
total mark, so a sub-question like 5.1 (one of five) may contribute up
to 6\% if evenly weighted.

\subsection{Task 1: Initial Processing
{[}10\%{]}}\label{task-1-initial-processing-10}

Weight within task:

\begin{itemize}
\tightlist
\item
  1.1 Extraction and Restructuring (50\%)
\item
  1.2 Conversion and Summarisation (50\%)
\end{itemize}

Rubric:

\begin{itemize}
\tightlist
\item
  Technical Accuracy (50\%)

  \begin{itemize}
  \tightlist
  \item
    Correct unpacking of NetCDF variables (names, dimensionality): 15\%
  \item
    Time conversion handled correctly (POSIX timestamps): 10\%
  \item
    Data reshaped into appropriate long format: 15\%
  \item
    Presence of appropriate columns (year, quarter, etc.): 10\%
  \end{itemize}
\item
  Depth of Analysis (20\%)

  \begin{itemize}
  \tightlist
  \item
    Efficient use of methods (e.g.~hyper\_tibble() or expand.grid() vs
    brute loops): 10\%
  \item
    Use of Cartesian indexing or equivalent vectorised operation: 10\%
  \end{itemize}
\item
  Clarity and Communication (20\%)

  \begin{itemize}
  \tightlist
  \item
    Code is readable, well-commented: 10\%
  \item
    Summary of the resulting data structure shown and interpretable:
    10\%
  \end{itemize}
\item
  Critical Thinking (10\%)

  \begin{itemize}
  \tightlist
  \item
    Indicates understanding of spatial × temporal structure and mentions
    NA implications: 10\%
  \end{itemize}
\end{itemize}

\subsection{Task 2: Exploratory Data Analysis
{[}10\%{]}}\label{task-2-exploratory-data-analysis-10}

2.1 Weighted Mean Time Series

\begin{itemize}
\tightlist
\item
  \begin{enumerate}
  \def\labelenumi{(\arabic{enumi})}
  \tightlist
  \item
    Weighted mean across time: 15\%
  \end{enumerate}
\item
  \begin{enumerate}
  \def\labelenumi{(\arabic{enumi})}
  \setcounter{enumi}{1}
  \tightlist
  \item
    Time series for 100 pixels: 15\%
  \end{enumerate}
\end{itemize}

2.2 Summary Statistics:

\begin{itemize}
\tightlist
\item
  \begin{enumerate}
  \def\labelenumi{(\arabic{enumi})}
  \tightlist
  \item
    Descriptive stats: 20\%
  \end{enumerate}
\item
  \begin{enumerate}
  \def\labelenumi{(\arabic{enumi})}
  \setcounter{enumi}{1}
  \tightlist
  \item
    Visualisations: 20\%
  \end{enumerate}
\item
  \begin{enumerate}
  \def\labelenumi{(\arabic{enumi})}
  \setcounter{enumi}{2}
  \tightlist
  \item
    Interpretation: 20\%
  \end{enumerate}
\end{itemize}

2.3 Observation Density Map: 10\%

Rubric:

\begin{itemize}
\tightlist
\item
  Technical Accuracy (50\%)

  \begin{itemize}
  \tightlist
  \item
    Proper handling of weights and NA filtering: 10\%
  \item
    Correct aggregation logic (quarter, pixel, etc.): 10\%
  \item
    Appropriateness of visualisation syntax and ggplot conventions: 10\%
  \item
    Use of statistical descriptors (mean, sd, skew, etc.) correctly:
    10\%
  \item
    Map projection/geodesic coordinates and section overlay accuracy:
    10\%
  \end{itemize}
\item
  Depth of Analysis (20\%)

  \begin{itemize}
  \tightlist
  \item
    Commentary on skewness, kurtosis, and statistical implications: 10\%
  \item
    Recognition of seasonal/temporal signals in plots and stats: 10\%
  \end{itemize}
\item
  Clarity and Communication (20\%)

  \begin{itemize}
  \tightlist
  \item
    Plot labels, axes, titles intelligible and precise: 10\%
  \item
    Logical narrative supporting visualisations/statistics: 10\%
  \end{itemize}
\item
  Critical Thinking (10\%)

  \begin{itemize}
  \tightlist
  \item
    Justification of metric choices, handling of anomalous years: 5\%
  \item
    Suggestions of ecological explanations (e.g., photoperiod,
    storminess): 5\%
  \end{itemize}
\end{itemize}

\subsection{Task 3: Inferential Statistics I
{[}20\%{]}}\label{task-3-inferential-statistics-i-20}

Weight within task:

\begin{itemize}
\tightlist
\item
  \begin{enumerate}
  \def\labelenumi{(\arabic{enumi})}
  \tightlist
  \item
    Hypotheses: 10\%
  \end{enumerate}
\item
  \begin{enumerate}
  \def\labelenumi{(\arabic{enumi})}
  \setcounter{enumi}{1}
  \tightlist
  \item
    Model selection and justification: 20\%
  \end{enumerate}
\item
  \begin{enumerate}
  \def\labelenumi{(\arabic{enumi})}
  \setcounter{enumi}{2}
  \tightlist
  \item
    Assumption testing: 20\%
  \end{enumerate}
\item
  \begin{enumerate}
  \def\labelenumi{(\arabic{enumi})}
  \setcounter{enumi}{3}
  \tightlist
  \item
    Result interpretation and diagnostics: 50\%
  \end{enumerate}
\end{itemize}

Rubric:

\begin{itemize}
\tightlist
\item
  Technical Accuracy (50\%)

  \begin{itemize}
  \tightlist
  \item
    Correct use of linear model and specification (additive, no
    interaction): 20\%
  \item
    Explicit assumptions tested (normality, homogeneity): 10\%
  \item
    Proper model diagnostics and visual checks: 10\%
  \item
    Use of correct significance thresholds and p-value interpretation:
    10\%
  \end{itemize}
\item
  Depth of Analysis (20\%)

  \begin{itemize}
  \tightlist
  \item
    Justification for using aggregate means vs raw data: 10\%
  \item
    Consideration of alternative models (e.g., GAMs): 10\%
  \end{itemize}
\item
  Clarity and Communication (20\%)

  \begin{itemize}
  \tightlist
  \item
    Hypotheses stated cleanly, concisely: 10\%
  \item
    Figure/Table references integrated smoothly in the narrative: 10\%
  \end{itemize}
\item
  Critical Thinking (10\%)

  \begin{itemize}
  \tightlist
  \item
    Recognition of model limitations and implications (e.g.~low R²):
    10\%
  \end{itemize}
\end{itemize}

\subsection{Task 4: Spatial Assignment
{[}10\%{]}}\label{task-4-spatial-assignment-10}

4.1 Section Assignment: 5\%

4.2 Bioregion Assignment: 5\%

Rubric:

\begin{itemize}
\tightlist
\item
  Technical Accuracy (50\%)

  \begin{itemize}
  \tightlist
  \item
    Correct application of Haversine formula or great-circle logic: 20\%
  \item
    Accurate section\_id assignment: 10\%
  \item
    Bioregion mapping via join or merge: 10\%
  \item
    Correct data columns preserved/renamed: 10\%
  \end{itemize}
\item
  Depth of Analysis (20\%)

  \begin{itemize}
  \tightlist
  \item
    Efficiency of matching routine (e.g., mapply() or vectorised join):
    10\%
  \item
    Consideration of spatial boundaries (e.g., limiting to section
    1--22): 10\%
  \end{itemize}
\item
  Clarity and Communication (20\%)

  \begin{itemize}
  \tightlist
  \item
    Annotated code, explanation of proximity logic: 10\%
  \item
    Output (head(), summary(), tail()) shows assignment integrity: 10\%
  \end{itemize}
\item
  Critical Thinking (10\%)
\item
  Considers effect of section resolution or mapping error: 10\%
\end{itemize}

\subsection{Task 5: Inferential Statistics II
{[}30\%{]}}\label{task-5-inferential-statistics-ii-30}

Each sub-task contributes approximately 6\% unless reweighted
explicitly.

Rubric per sub-task (5.1--5.5):

\begin{itemize}
\tightlist
\item
  Technical Accuracy (50\%)

  \begin{itemize}
  \tightlist
  \item
    Model type (ANOVA, LM, ANCOVA) appropriate: 15\%
  \item
    Correct test execution (summary, diagnostics): 15\%
  \item
    Assumptions evaluated, violations addressed: 10\%
  \item
    Non-parametric alternative proposed when appropriate: 10\%
  \end{itemize}
\item
  Depth of Analysis (20\%)

  \begin{itemize}
  \tightlist
  \item
    Explicit rationale for model choice: 10\%
  \item
    Discussion of structure in data (nesting, lack of interaction): 10\%
  \end{itemize}
\item
  Clarity and Communication (20\%)

  \begin{itemize}
  \tightlist
  \item
    Hypotheses clearly and formally stated: 10\%
  \item
    Visualisations appropriately labelled and explained: 10\%
  \end{itemize}
\item
  Critical Thinking (10\%)

  \begin{itemize}
  \tightlist
  \item
    Insight into ecological implications of findings (e.g., BMP trend):
    10\%
  \end{itemize}
\end{itemize}

Add 1--2 bonus marks if:

\begin{itemize}
\tightlist
\item
  Multicollinearity (e.g., VIF) or autocorrelation (e.g., DW test) is
  discussed
\item
  Advanced diagnostics (e.g., Breusch--Pagan, TukeyHSD) are used
  correctly
\end{itemize}

\subsection{Task 6: Write-up {[}10\%{]}}\label{task-6-write-up-10}

Rubric:

\begin{itemize}
\tightlist
\item
  Technical Accuracy (50\%)

  \begin{itemize}
  \tightlist
  \item
    Consistent reference to previous results, correct figure/table
    interpretation: 25\%
  \item
    Accurate paraphrasing of statistical results: 15\%
  \item
    Adherence to 2-page length limit, integration of material: 10\%
  \end{itemize}
\item
  Depth of Analysis (20\%)

  \begin{itemize}
  \tightlist
  \item
    Rich synthesis across Tasks 2--5, not isolated repetition: 10\%
  \item
    Conceptual connection of seasonality, trend, and spatial
    heterogeneity: 10\%
  \end{itemize}
\item
  Clarity and Communication (20\%)

  \begin{itemize}
  \tightlist
  \item
    Coherent scientific writing style, flowing paragraph structure: 10\%
  \item
    Effective integration of figure references and literature: 10\%
  \end{itemize}
\item
  Critical Thinking (10\%)

  \begin{itemize}
  \tightlist
  \item
    Limitations clearly acknowledged and reflected on: 5\%
  \item
    Forward-looking ecological insight or recommendation offered: 5\%
  \end{itemize}
\end{itemize}




\end{document}
