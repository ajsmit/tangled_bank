\documentclass[a4paper,10pt]{article}
\usepackage{graphicx}
\usepackage{array}
\usepackage{booktabs}
\usepackage{longtable}
\usepackage{geometry}
\usepackage[raggedright]{ragged2e}
\geometry{left=1in, right=1in, top=1in, bottom=1in}

% --- Custom Column Type (using ragged2e) ---

\newcolumntype{R}[1]{>{\RaggedRight}p{#1}}

\usepackage{fontspec}         % Allows font specification
\setmainfont[
  Renderer=Basic, % Recommended for plain text
  UprightFont = MinionPro-Regular,
  ItalicFont = MinionPro-It,
  ItalicFeatures = { SmallCapsFont = MinionPro-It },
  SlantedFont = MinionPro-Regular,
  SlantedFeatures= { FakeSlant=0.2 },
  BoldFont = MinionPro-Bold,
  BoldFeatures = { SmallCapsFont = MinionPro-Bold },
  BoldItalicFont = MinionPro-BoldIt,
  BoldItalicFeatures = { SmallCapsFont = MinionPro-BoldIt },
  BoldSlantedFont= MinionPro-Bold,
  BoldSlantedFeatures= { FakeSlant=0.2, SmallCapsFont = MinionPro-Bold },
  SmallCapsFont = MinionPro-Regular,
  SmallCapsFeatures={ RawFeature=+smcp },
  Ligatures=TeX,
  Numbers={OldStyle, Proportional}
]{MinionPro-Regular}

% --- Other Typography Settings ---

\frenchspacing                % Single space after periods
\tolerance=400                % Default is 200; higher values allow more relaxed line-breaking.
\emergencystretch=3em         % Adds additional space to help line-breaking.
\hyphenpenalty=20             % Default is 50; lower values encourage hyphenation.

% --- Microtype Settings (adjust only if needed) ---

\usepackage{microtype}        % Improves typography
\microtypesetup{
   tracking = true,
   protrusion=true,
   expansion=true,
   factor = 1100,
   stretch = 15,
   shrink = 15
}

\title{\textbf{BCB744 Daily Self-Assessment Tasks: Rubric}}
\author{}
\date{}

\begin{document}
\maketitle

\section*{Student Name: \underline{\hspace{5cm}} Task: \underline{\hspace{5cm}}}

\section*{Daily Self-Assessment Tasks}
BCB744 (Introduction to R, and Biostatistics) relies on the expectation that you will engage in regular, honest self-reflection about your grasp of each day’s lecture content. The honesty of these reflections cannot be overstated: each task should be rated according to the mark allocation next to the criteria below.
These self-assessment marks will be kept on record and serve as an indicator of your progress.
\vspace{12pt}

\noindent Except for random checks to see if your self-assigned mark reflects reality (as judged by myself or a teaching assistant for how closely it aligns with the model answers), they will not be assessed. We will also discourage students from undertaking the Intro R Test and the Biostatistics Test if their self-assessment scores are consistently low.
\vspace{12pt}

\noindent For the Daily Self-Assessment Tasks to be effective, you must work alone on all of them.
\vspace{12pt}

\noindent All the tasks must be submitted by the deadline.

\section*{Assessment Criteria}
When assessing the tasks, pay attention to the following criteria:

{\small
\begin{longtable}{R{3.5cm} R{8.5cm} R{3cm}}
    \toprule
    \textbf{Criterion} & \textbf{Description} & \textbf{Marks Allocated} \\
    \midrule
    \textbf{Content} & Questions answered in order. Annotations included (metadata in file header, comments about code, ideas, and approach). & 10 \% \\
    \textbf{Code Formatting \& Presentation} & Application of R code conventions (e.g., spaces around \texttt{<-}, after \texttt{\#}, after commas; correct use of new lines for each \texttt{dplyr} function (lines end in \texttt{\%>\%}) or \texttt{ggplot} layer (lines end in \texttt{+}), etc.). Logical structure of document and thorough, consistent use of headings and subheadings. Appearance and attention to detail. Code runs correctly without errors. & 40 \% \\
    \textbf{Correct Answers} & Demonstrating the correctness of answers by comparing them to the model answers. Correctly labelled, publication quality figures. & 50 \% \\
    \bottomrule
\end{longtable}}

\section*{Final Score: \underline{\hspace{2cm}} / 100\%}

\end{document}
