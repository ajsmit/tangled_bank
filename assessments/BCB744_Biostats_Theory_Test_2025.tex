% Options for packages loaded elsewhere
% Options for packages loaded elsewhere
\PassOptionsToPackage{unicode}{hyperref}
\PassOptionsToPackage{hyphens}{url}
\PassOptionsToPackage{dvipsnames,svgnames,x11names}{xcolor}
%
\documentclass[
  10t,
]{article}
\usepackage{xcolor}
\usepackage{amsmath,amssymb}
\setcounter{secnumdepth}{-\maxdimen} % remove section numbering
\usepackage{iftex}
\ifPDFTeX
  \usepackage[T1]{fontenc}
  \usepackage[utf8]{inputenc}
  \usepackage{textcomp} % provide euro and other symbols
\else % if luatex or xetex
  \usepackage{unicode-math} % this also loads fontspec
  \defaultfontfeatures{Scale=MatchLowercase}
  \defaultfontfeatures[\rmfamily]{Ligatures=TeX,Scale=1}
\fi
\usepackage{lmodern}
\ifPDFTeX\else
  % xetex/luatex font selection
\fi
% Use upquote if available, for straight quotes in verbatim environments
\IfFileExists{upquote.sty}{\usepackage{upquote}}{}
\IfFileExists{microtype.sty}{% use microtype if available
  \usepackage[]{microtype}
  \UseMicrotypeSet[protrusion]{basicmath} % disable protrusion for tt fonts
}{}
\makeatletter
\@ifundefined{KOMAClassName}{% if non-KOMA class
  \IfFileExists{parskip.sty}{%
    \usepackage{parskip}
  }{% else
    \setlength{\parindent}{0pt}
    \setlength{\parskip}{6pt plus 2pt minus 1pt}}
}{% if KOMA class
  \KOMAoptions{parskip=half}}
\makeatother
% Make \paragraph and \subparagraph free-standing
\makeatletter
\ifx\paragraph\undefined\else
  \let\oldparagraph\paragraph
  \renewcommand{\paragraph}{
    \@ifstar
      \xxxParagraphStar
      \xxxParagraphNoStar
  }
  \newcommand{\xxxParagraphStar}[1]{\oldparagraph*{#1}\mbox{}}
  \newcommand{\xxxParagraphNoStar}[1]{\oldparagraph{#1}\mbox{}}
\fi
\ifx\subparagraph\undefined\else
  \let\oldsubparagraph\subparagraph
  \renewcommand{\subparagraph}{
    \@ifstar
      \xxxSubParagraphStar
      \xxxSubParagraphNoStar
  }
  \newcommand{\xxxSubParagraphStar}[1]{\oldsubparagraph*{#1}\mbox{}}
  \newcommand{\xxxSubParagraphNoStar}[1]{\oldsubparagraph{#1}\mbox{}}
\fi
\makeatother


\usepackage{longtable,booktabs,array}
\usepackage{calc} % for calculating minipage widths
% Correct order of tables after \paragraph or \subparagraph
\usepackage{etoolbox}
\makeatletter
\patchcmd\longtable{\par}{\if@noskipsec\mbox{}\fi\par}{}{}
\makeatother
% Allow footnotes in longtable head/foot
\IfFileExists{footnotehyper.sty}{\usepackage{footnotehyper}}{\usepackage{footnote}}
\makesavenoteenv{longtable}
\usepackage{graphicx}
\makeatletter
\newsavebox\pandoc@box
\newcommand*\pandocbounded[1]{% scales image to fit in text height/width
  \sbox\pandoc@box{#1}%
  \Gscale@div\@tempa{\textheight}{\dimexpr\ht\pandoc@box+\dp\pandoc@box\relax}%
  \Gscale@div\@tempb{\linewidth}{\wd\pandoc@box}%
  \ifdim\@tempb\p@<\@tempa\p@\let\@tempa\@tempb\fi% select the smaller of both
  \ifdim\@tempa\p@<\p@\scalebox{\@tempa}{\usebox\pandoc@box}%
  \else\usebox{\pandoc@box}%
  \fi%
}
% Set default figure placement to htbp
\def\fps@figure{htbp}
\makeatother





\setlength{\emergencystretch}{3em} % prevent overfull lines

\providecommand{\tightlist}{%
  \setlength{\itemsep}{0pt}\setlength{\parskip}{0pt}}



 


% preamble.tex

% --- Document Structure and Layout ---

\usepackage[a4paper, total={6in, 8in}]{geometry}

% --- Paragraph settings ---

\setlength{\parindent}{0pt}
\setlength{\parskip}{6pt}

% --- Color Definitions ---

\usepackage[x11names]{xcolor} % Required for specifying custom colors, load before tcolorbox
\definecolor{headingblue}{RGB}{23,48,191}
\definecolor{boxtitle}{HTML}{F0F4F8}
\definecolor{boxbody}{HTML}{FBFDFF}
\definecolor{mainboxframe}{HTML}{F0F4F8}
\definecolor{subboxframe}{HTML}{F0F4F8}
\definecolor{crimson}{HTML}{880000}
\definecolor{monocolor}{RGB}{64,224,208}

% --- Fonts and Encoding ---

\usepackage{fontspec}         % Allows font specification
% \usepackage{amsmath}          % For math symbols

% Main Font
\setmainfont[
  UprightFont = *-Regular,
  ItalicFont = *-Italic,
  ItalicFeatures = { SmallCapsFont = *-Italic },
  SlantedFont = *-Regular,
  SlantedFeatures= { FakeSlant=0.2 },
  BoldFont = *-Bold,
  BoldFeatures = { SmallCapsFont = *-Bold },
  BoldItalicFont = *-BoldItalic,
  BoldItalicFeatures = { SmallCapsFont = *-BoldItalic },
  BoldSlantedFont= *-Bold,
  BoldSlantedFeatures= { FakeSlant=0.2, SmallCapsFont = *-Bold },
  SmallCapsFont = *-Regular,
  SmallCapsFeatures={ RawFeature=+smcp },
  Ligatures=TeX,
  Numbers={OldStyle, Proportional}
]{StixTwoText}

% Math Font
\setmathfont{StixTwoMath.otf}

% Monospace Font
\setmonofont[
  Scale=0.84
]{FiraCode Nerd Font}
\renewcommand{\ttfamily}{\small\fontspec{FiraCode Nerd Font}\color{DeepSkyBlue4}}
\renewcommand{\texttt}[1]{{\ttfamily #1}}

% --- Packages for Tables ---

\usepackage{array}            % For table column width specification
\usepackage{booktabs}         % For table rules
\usepackage{ragged2e}         % For text alignment (used with \newcolumntype)

% --- Headers and Footers ---

\usepackage{fancyhdr}
\pagestyle{fancy}
\renewcommand{\sectionmark}[1]{\markright{#1}{}}
\fancyhf{}
\fancyhead[LE,RO]{\thepage}
\fancyhead[LO]{\textsc{\MakeLowercase{\leftmark}}}
\fancyhead[RE]{\textsc{\MakeLowercase{\rightmark}}}

% --- Other Packages ---

\usepackage[version=4]{mhchem}% Formula subscripts using \ce{}

% --- Key Terms

\newcommand{\keyterm}[1]{\textsc{#1}}

%% Create a command for color emphasis
\newcommand{\highlight}[1]{\textcolor{crimson}{#1}}

% --- Boxes ---

% Define the mdframed environment
\usepackage{float}
\usepackage{mdframed}

% 1) Define a new float environment called "boxfloat"
\newfloat{boxfloat}{htbp}{lob}
\floatname{boxfloat}{Box}

% 3) Define the environment that wraps mdframed in a float
\newenvironment{boxedfloat}[2][]{%
  % Advance the box counter to produce "Chapter.BoxNo"
  \refstepcounter{boxcounter}%
  % Begin the float environment
  \begin{boxfloat}[htbp]
  % Begin the mdframed styling
  \begin{mdframed}[
    backgroundcolor=gray!5,
    innertopmargin=6pt,
    innerbottommargin=6pt,
    innerrightmargin=6pt,
    innerleftmargin=6pt,
    linewidth=0.25pt,
    linecolor=black,
    roundcorner=8pt,    % or 0pt if you prefer sharp corners
    skipabove=12pt,     % vertical space above the box
    skipbelow=12pt,     % vertical space below the box
    innermargin=0pt,
    outermargin=0pt
  ]%
    % Typeset the box heading: "Box 1.1. My Title"
    \setlength{\parindent}{0em}%
    \setlength{\parskip}{3pt}%
    \RaggedRight
    % Both "Box" and the user-supplied title are in small caps
    \small% switch the box contents to smaller text
    {\scshape Box \theboxcounter. #2}\par
    \vspace{6pt} % a little space after the heading
}{%
    \end{mdframed}
    \end{boxfloat}
}

% --- sansblock Environment ---

\usepackage{sourcesanspro}    % Load Source Sans Pro
\setsansfont{Source Sans Pro} % Set it as the sans-serif font
\newenvironment{sansblock}[1]
    {\small\sffamily\raggedright{\scshape #1}\ } % Ensure small caps for the title
  {} % End environment: no special commands needed

% --- Custom Column Type (using ragged2e) ---

\newcolumntype{R}[1]{>{\RaggedRight}p{#1}}

% ---  Margin Notes ---

\usepackage{marginnote}
\renewcommand*{\marginfont}{\footnotesize\itshape} % Style for margin notes

%% Set margin note outer margin to 0.7in
\setlength{\marginparwidth}{1.25in}

% --- Epigraph ---

\usepackage{epigraph}
\setlength\epigraphwidth{.9\textwidth}
\newenvironment{quotepara}
  {\itshape\raggedright\small\setlength{\parskip}{0.5em}} % Add small space between paragraphs
  {}
\renewcommand{\textflush}{quotepara}

%% Define a new epigraph environment without the horizontal rule and source/author
\newenvironment{simpleepigraph}
  {\begin{list}{}%
      {\setlength{\leftmargin}{2em}% Left margin
       \setlength{\rightmargin}{2em}% Right margin
       \setlength{\topsep}{1em}% Space above the epigraph
       \setlength{\itemsep}{0pt}% Space between items (irrelevant here)
       \setlength{\parsep}{0pt}}% Space between paragraphs
   \item\relax\raggedright\small} % Apply ragged-right and italic style for the epigraph text
  {\end{list}}

% --- Small Caps ---

\newcommand{\flatcaps}[1]{\textsc{\MakeLowercase{#1}}}

% --- Lists ---

%% General settings for all lists
\usepackage{enumitem}

% Global settings following Bringhurst's principles
% A global default to keep lists tight, but still allow subtle spacing:
\setlist{
  nosep,         % No extra space between items
  topsep=0.6em,  % A bit of space before/after the list
  parsep=0pt,
  partopsep=0pt
}

% First-level itemize (unordered) lists:
\setlist[itemize,1]{
  label=\textbullet,
  labelsep=0.4em,        % Space from bullet to text
  labelwidth=1em,        % Horizontal space set aside for bullet
  leftmargin=\dimexpr 1em + 0.4em\relax,
  itemindent=0pt,
  listparindent=0pt,
  align=left
}

% Second-level itemize, with a subtler symbol:
\setlist[itemize,2]{
  label=--,
  labelsep=0.4em,
  labelwidth=1em,
  leftmargin=\dimexpr 1em + 0.4em\relax,
  itemindent=0pt,
  listparindent=0pt,
  align=left
}

% First-level enumerate (ordered) lists:
\setlist[enumerate,1]{
  label=\arabic*.,
  labelsep=0.4em,
  labelwidth=1em,
  leftmargin=\dimexpr 1em + 0.4em\relax,
  itemindent=0pt,
  listparindent=0pt,
  align=left
}

% Second-level enumerate (letters, or you could do roman numerals):
\setlist[enumerate,2]{
  label=\alph*.,
  labelsep=0.4em,
  labelwidth=1em,
  leftmargin=\dimexpr 1em + 0.4em\relax,
  itemindent=0pt,
  listparindent=0pt,
  align=left
}

% --- Custom Chapter/Section Styles ---

\usepackage[compact]{titlesec} % Allows creating custom chapter styles
\titleformat{\chapter}[display]
  {\fontsize{60}{62}\bfseries}
  {\thechapter}
  {0pt}
  {\huge\noindent}
\titlespacing*{\chapter}{0pt}{0pt}{40pt}

\titleformat{\section}
  {\normalsize\normalfont}
  {\thesection}
  {0.6em}
  {\flatcaps}
\titlespacing*{\section}{0pt}{1\baselineskip}{1\baselineskip}

\titleformat{\subsection}[block]
  {\normalsize\normalfont} % defines the font size and style for the entire subsection heading, including both the number and the title
  {\thesubsection} % defines the format of the subsection number
  {1em} % the horizontal space between the subsection number and the title
  {\itshape} % defines the format of the subsection title itself
\titlespacing*{\subsection}{0pt}{1\baselineskip}{1\baselineskip}

\titleformat{\subsubsection}[runin]
  {\normalsize\normalfont} % defines the font size and style for the entire subsection heading, including both the number and the title
  {\thesubsubsection} % defines the format of the subsection number
  {1em} % the horizontal space between the subsection number and the title
  {\itshape}[.] % defines the format of the subsection title itself
\titlespacing*{\subsubsection}{0pt}{1\baselineskip}{1\baselineskip}

\titleformat{\paragraph}[runin]
  {\flatcaps}
  {\theparagraph}
  {0pt}
  {}

% --- Footnotes ---

\usepackage[norule,ragged,hang]{footmisc}  % Load footmisc with ragged option
\renewcommand{\footnotelayout}{\RaggedRight\footnotesize} % Typeset footnotes in \RaggedRight
\setlength{\footnotemargin}{1.5em}    % Adjust space between number and text
\makeatletter
\renewcommand{\@makefntext}[1]{%
    \parindent 1em%                    Set parindent for footnote text
    \noindent
    \hb@xt@ 1.8em{%                   Set hanging indent for footnote text
        \hss\@thefnmark.%
    }
    \RaggedRight #1%                 Typeset footnote text ragged right
}
\makeatother

% --- Captions ---

\usepackage{caption}
\captionsetup{
  font={small},
  labelfont={bf},
  textfont={},
  width=0.9\textwidth,
  justification=justified,
  labelformat=default,
  labelsep=period,
  format=plain
}
\renewcommand{\captionlabelfont}{\bfseries\scshape}

% --- Hyperlinks ---

\usepackage{hyperref}           % Load after most other packages, but before cleveref
\hypersetup{
    colorlinks=true,
    linkcolor=blue,
    filecolor=magenta,
    urlcolor=cyan,
    pdftitle={Overleaf Example},
    pdfpagemode=FullScreen,
    }

\urlstyle{same}

% --- Miscellaneous ---

%% Define a new command to print the current page number to the console
\ifluatex
  \usepackage{luacode}
  \usepackage{shellesc}
  \newcommand{\printpagenumber}{%
    \directlua{
      local pagenumber = tex.count.page
      print(string.format("Currently processing page: %d", pagenumber))
    }
  }
\fi

\usepackage{etoolbox}          % General package for patching commands
\usepackage{iftex}             % Detects the engine used
\usepackage{ellipsis}          % Fixes spacing around ellipses
\AddToHook{env/Highlighting/begin}{\small} % Set the code chunk font size globally

%% Use lining fonts
\newcommand\lining{\addfontfeatures{Numbers={Monospaced, Lining}}}
\AtBeginEnvironment{tabular}{\lining} % In tables
\renewcommand{\theequation}{ {\lining\arabic{equation}}} % For equation numbers

% --- Index (if needed) ---

\usepackage{makeidx}
\makeindex

% --- Other Typography Settings ---

\frenchspacing                % Single space after periods
\tolerance=400                % Default is 200; higher values allow more relaxed line-breaking.
\emergencystretch=3em         % Adds additional space to help line-breaking.
\hyphenpenalty=20             % Default is 50; lower values encourage hyphenation.

% --- Microtype Settings (adjust only if needed) ---

\usepackage{microtype}        % Improves typography
\microtypesetup{
   tracking = true,
   protrusion=true,
   expansion=true,
   factor = 1100,
   stretch = 15,
   shrink = 15
}

\makeatletter
\@ifpackageloaded{caption}{}{\usepackage{caption}}
\AtBeginDocument{%
\ifdefined\contentsname
  \renewcommand*\contentsname{Table of contents}
\else
  \newcommand\contentsname{Table of contents}
\fi
\ifdefined\listfigurename
  \renewcommand*\listfigurename{List of Figures}
\else
  \newcommand\listfigurename{List of Figures}
\fi
\ifdefined\listtablename
  \renewcommand*\listtablename{List of Tables}
\else
  \newcommand\listtablename{List of Tables}
\fi
\ifdefined\figurename
  \renewcommand*\figurename{Figure}
\else
  \newcommand\figurename{Figure}
\fi
\ifdefined\tablename
  \renewcommand*\tablename{Table}
\else
  \newcommand\tablename{Table}
\fi
}
\@ifpackageloaded{float}{}{\usepackage{float}}
\floatstyle{ruled}
\@ifundefined{c@chapter}{\newfloat{codelisting}{h}{lop}}{\newfloat{codelisting}{h}{lop}[chapter]}
\floatname{codelisting}{Listing}
\newcommand*\listoflistings{\listof{codelisting}{List of Listings}}
\makeatother
\makeatletter
\makeatother
\makeatletter
\@ifpackageloaded{caption}{}{\usepackage{caption}}
\@ifpackageloaded{subcaption}{}{\usepackage{subcaption}}
\makeatother
\makeatletter
\@ifpackageloaded{sidenotes}{}{\usepackage{sidenotes}}
\@ifpackageloaded{marginnote}{}{\usepackage{marginnote}}
\makeatother
\usepackage{bookmark}
\IfFileExists{xurl.sty}{\usepackage{xurl}}{} % add URL line breaks if available
\urlstyle{same}
\hypersetup{
  pdftitle={BCB744: Biostatistics R Test},
  pdfauthor={Smit, A. J.},
  colorlinks=true,
  linkcolor={blue},
  filecolor={blue},
  citecolor={blue},
  urlcolor={blue},
  pdfcreator={LaTeX via pandoc}}


\title{BCB744: Biostatistics R Test}
\author{Smit, A. J.}
\date{2025-04-25}
\begin{document}
\maketitle


\section{About the Test}\label{about-the-test}

The Biostatistics Test will start at 8:30 on 25 April, 2025 and you have
until 11:30 to complete it. This is the Theory Test, which must be
conducted on campus. The theory component contributes 30\% of the final
assessment marks.

\section{Assessment Policy}\label{assessment-policy}

{The marks indicated for each section reflect the relative weight (and
hence depth expected in your response) rather than a rigid check-list of
individual points.} Your answers should demonstrate a comprehensive
understanding of the concepts and techniques required. Higher marks will
be awarded for narratives that demonstrate not only conceptual and
theoretical correctness but also insightful discussion and clear
communication of insights or findings. We are assessing your ability to
think systematically through complex inquiries, make appropriate
theoretical and methodological choices, and present feedback in a
coherent narrative that reveals deep understanding.

Please refer to the
\href{https://tangledbank.netlify.app/BCB744/BCB744_index.html\#sec-policy}{Assessment
Policy} for more information on the test format and rules.

\section{Theory Test}\label{theory-test}

{\textbf{This is the closed book assessment.}}

Below is a set of questions to answer. You must answer all questions in
the allocated time of 3-hr. Please write your answers in a neatly
formatted Word document and submit it to the iKamva platform.

Clearly indicate the question number and provide detailed explanations
for your answers. Use Word's headings and subheadings facility to
structure your document logically.

Naming convention:
\texttt{Biostatistics\_Theory\_Test\_YourSurname.docx}

\subsection{Question 1}\label{question-1}

Imagine you are presented with the following five research scenarios
(see below). In each case, your task is to decide which statistical
method would be most appropriate and to justify your reasoning.

For each of the five scenarios below:

\begin{enumerate}
\def\labelenumi{\alph{enumi}.}
\tightlist
\item
  Identify the appropriate statistical method.
\item
  Explain why this method is more suitable than the others listed.
\item
  Clearly identify the dependent and independent variables (where
  applicable), and describe their type (categorical, continuous, etc.).
\item
  Describe what the method would allow you to infer, and what its
  limitations might be in the given context.
\end{enumerate}

Scenarios:

\begin{enumerate}
\def\labelenumi{\arabic{enumi}.}
\tightlist
\item
  A researcher wants to compare average leaf nitrogen content between
  two plant species growing in the same habitat.
\item
  An ecologist is interested in whether water temperature predicts fish
  body size across multiple river sites.
\item
  A conservation biologist is comparing average bird abundance across
  five habitat types, while also accounting for altitude which is known
  to influence bird detection rates.
\item
  A physiologist wants to explore whether heart rate and body
  temperature are linearly associated in a sample of animals under heat
  stress conditions.
\item
  A botanist tests whether fertiliser type (3 levels: organic,
  inorganic, control) affects plant height, but only has access to a
  small sample from each group.
\end{enumerate}

\textbf{{[}20 marks{]}}

\textbf{Answer}

Scenario 1:

\begin{enumerate}
\def\labelenumi{\alph{enumi}.}
\tightlist
\item
  Independent (two-sample) \emph{t}-test (or Mann-Whitney U test if data
  are not normally distributed).
\item
  The \emph{t}-test is appropriate for comparing means between two
  independent groups (species). The Mann-Whitney U test is a
  non-parametric alternative that does not assume normality.
\item
  Dependent variable: leaf nitrogen content (continuous); independent
  variable: plant species (categorical).
\item
  The \emph{t}-test allows for inference about differences in means, but
  is sensitive to normality and equal variance assumptions. The
  Mann-Whitney U test is less sensitive to these assumptions but does
  not provide mean differences (differences based on ranks).
\end{enumerate}

Scenario 2:

\begin{enumerate}
\def\labelenumi{\alph{enumi}.}
\tightlist
\item
  Linear regression analysis.
\item
  Linear regression is suitable for assessing the relationship between a
  continuous dependent variable (fish body size) and a continuous
  independent variable (water temperature).
\item
  Dependent variable: fish body size (continuous); independent variable:
  water temperature (continuous).
\item
  Linear regression allows for inference about the strength and
  direction of the relationship, but assumes linearity and
  homoscedasticity. It may not capture non-linear relationships or
  interactions.
\end{enumerate}

Scenario 3:

\begin{enumerate}
\def\labelenumi{\alph{enumi}.}
\tightlist
\item
  Analysis of covariance (ANCOVA).
\item
  ANCOVA is appropriate for comparing means across multiple groups
  (habitat types) while controlling for a covariate (altitude).
\item
  Dependent variable: bird abundance (continuous); independent variable:
  habitat type (categorical); covariate: altitude (continuous).
\item
  ANCOVA allows for inference about group differences while accounting
  for the influence of altitude, but assumes homogeneity of regression
  slopes and normality of residuals.
\end{enumerate}

Scenario 4:

\begin{enumerate}
\def\labelenumi{\alph{enumi}.}
\tightlist
\item
  Linear regression analysis.
\item
  Linear regression is suitable for exploring the relationship (often
  causal) between two continuous variables (heart rate and body
  temperature).
\item
  Dependent variable: heart rate (continuous); independent variable:
  body temperature (continuous).
\item
  Linear regression allows for inference about the strength and
  direction of the relationship, but assumes linearity and
  homoscedasticity. It may not capture non-linear relationships or
  interactions.
\end{enumerate}

Scenario 5:

\begin{enumerate}
\def\labelenumi{\alph{enumi}.}
\tightlist
\item
  One-way ANOVA (or Kruskal-Wallis test if data are not normally
  distributed).
\item
  One-way ANOVA is appropriate for comparing means across three or more
  independent groups (fertiliser types). The Kruskal-Wallis test is a
  non-parametric alternative that does not assume normality.
\item
  Dependent variable: plant height (continuous); independent variable:
  fertiliser type (categorical).
\item
  One-way ANOVA allows for inference about differences in means across
  groups, but assumes normality and homogeneity of variances. The
  Kruskal-Wallis test is less sensitive to these assumptions but does
  not provide mean differences (differences based on ranks).
\end{enumerate}

\begin{small}
\begin{raggedright}
\begin{longtable*}{|p{4.2cm}|p{9cm}|p{1.8cm}|}
\hline
\textbf{Assessment Criterion} & \textbf{Descriptor} & \textbf{Marks} \\
\hline
\textbf{1. Conceptual grasp of epistemology} & 
Shows understanding of epistemology as a theory of knowledge: how we know, not just what we know. Distinguishes epistemic claims from metaphysical or moral ones. \newline
\textit{0–1:} Misunderstands or omits the concept. \newline
\textit{2–3:} Partial understanding; conflates with method or belief. \newline
\textit{4–5:} Clearly articulates epistemology in context; shows reflective engagement. & 
0–5 \\
\hline
\textbf{2. Explanation of scientific epistemic structure} & 
Demonstrates understanding of how the scientific method justifies knowledge: e.g., empirical observation, theory-laden verification, replication, falsifiability, and provisionality. \newline
\textit{0–1:} Provides no or incorrect explanation. \newline
\textit{2–3:} References features like evidence or experiments, but lacks structure. \newline
\textit{4–5:} Offers a coherent account of how science generates and revises claims. & 
0–5 \\
\hline
\textbf{3. Contrast with faith-based systems} & 
Identifies how religious or mystical traditions legitimise knowledge through authority, revelation, or inner conviction. Avoids caricature or simplification. \newline
\textit{0–1:} Simplistic binary (e.g., “science = truth, religion = belief”). \newline
\textit{2–3:} Describes basic contrast but misses nuance. \newline
\textit{4–5:} Analyses contrasts in justification, verification, and correction. & 
0–5 \\
\hline
\textbf{4. Integration of material from assigned reading} & 
Effectively engages with relevant material from the assigned chapter (e.g., Galileo’s telescope, instrument-based trust, constructivist critiques). \newline
\textit{0:} No reference to or engagement with the text. \newline
\textit{1:} Superficial mention without integration. \newline
\textit{2–3:} Clear synthesis of reading into argument. & 
0–3 \\
\hline
\textbf{5. Quality of argumentation and writing} & 
Clarity, structure, and originality of response. Logical progression, precise language, and effective use of examples. \newline
\textit{0:} Incoherent or poorly expressed. \newline
\textit{1:} Reasonably clear with occasional lapses. \newline
\textit{2:} Fluent, well-organised, and compelling. & 
0–2 \\
\hline
\multicolumn{2}{|r|}{\textbf{Total}} & \textbf{20} \\
\hline
\end{longtable*}
\end{raggedright}
\end{small}

\subsection{Question 2}\label{question-2}

Science does not rely on certainty but on scepticism and structured
doubt. Its premise is not the claim to final truth; rather, it has the
capacity to generate reliable, revisable knowledge through empirical
observation, theoretical coherence, and methodological transparency.

In contrast, faith-based systems appeal to revelation, authority, or
moral intuition -- forms of conviction that do not invite or value
independent verification. Yet both systems organise trust. What, then,
distinguishes scientific knowledge from belief? What makes the
scientific method a unique epistemological endeavour?

\textbf{Question:} What is the basis of knowledge in the scientific
method, and how does this differ from the basis of knowledge in
faith-based systems such as religion or mysticism? In your answer,
consider the roles of observation, verification, theoretical coherence,
and error correction in scientific reasoning, and contrast these with
how knowledge is ``made real'' in non-empirical approaches.

\textbf{{[}15 marks{]}}

\textbf{Answer}

\begin{small}
\begin{raggedright}
\begin{longtable*}{|p{4.2cm}|p{9cm}|p{1.8cm}|}
\hline
\textbf{Assessment Criterion} & \textbf{Descriptor} & \textbf{Marks} \\
\hline
\textbf{1. Epistemological basis of scientific method} & 
Identifies how science generates and legitimises knowledge through observation, verification, coherence, error correction, and structured doubt. \newline
\textit{0–1:} Fails to explain or conflates epistemology with method or belief. \newline
\textit{2–3:} Some understanding of empirical structure, but lacks clarity or depth. \newline
\textit{4:} Coherent account of science’s epistemological architecture. & 
0–4 \\
\hline
\textbf{2. Contrast with faith-based epistemologies} & 
Explains how belief systems such as religion or mysticism ground knowledge in non-empirical sources (revelation, authority, moral intuition). \newline
\textit{0–1:} Simplistic or dismissive contrast. \newline
\textit{2:} Recognises distinction but lacks detail or nuance. \newline
\textit{3:} Articulates key contrasts in verification, justification, and trust. & 
0–3 \\
\hline
\textbf{3. Use of key terms and concepts} & 
Employs terms such as observation, verification, coherence, falsifiability, and “made real” in epistemically meaningful ways. \newline
\textit{0–1:} Little or no use of relevant concepts. \newline
\textit{2:} Some terminology used but inconsistently or unclearly. \newline
\textit{3:} Accurate and conceptually integrated use of language. & 
0–3 \\
\hline
\textbf{4. Comparative insight and originality} & 
Shows insight into how both systems organise trust and distinguish belief from knowledge. Avoids binary clichés. \newline
\textit{0–1:} Uncritical or overly oppositional. \newline
\textit{2:} Reasonable contrast, but surface-level. \newline
\textit{3:} Offers reflective or original comparison of epistemic norms. & 
0–3 \\
\hline
\textbf{5. Coherence and written expression} & 
Organised, precise, and cogent writing; ideas flow logically. \newline
\textit{0:} Poorly expressed or incoherent. \newline
\textit{1:} Understandable, but uneven. \newline
\textit{2:} Clear, structured, and engaging. & 
0–2 \\
\hline
\multicolumn{2}{|r|}{\textbf{Total}} & \textbf{15} \\
\hline
\end{longtable*}
\end{raggedright}
\end{small}

\subsection{Question 3}\label{question-3}

Throughout history, the development of statistical reasoning has been
shaped not just by mathematical discoveries, but by synergies across
intellectual traditions, technological innovation, and societal
imperatives. From ancient record-keeping and proto-quantification,
through the epistemic insights of the Renaissance and Enlightenment, to
the formalisation of probabilistic thinking, statistics has evolved
alongside shifting ideas about what it means to \emph{know}, to
\emph{measure}, and to \emph{infer}.

\textbf{Question:} How have historical interactions between these forces
-- ideas, instruments, and institutions -- shaped the philosophy
underpinning statistical practice as we know it today? In your response,
identify and critically examine what you consider, with justification,
to be five major conceptual or methodological turning points. These may
include developments in logical reasoning, technological breakthroughs
that extended observational capacity, institutional needs for
demographic governance, or shifts in philosophical approaches to
uncertainty and knowledge.

Your analysis should not simply recount historical facts, but provide a
reasoned argument about how each moment contributed to the emergence of
statistics as a knowledge framework -- that is, not just a set of
techniques, but a way of thinking about the world.

\textbf{{[}20 marks{]}}

\textbf{Answer}

\begin{small}
\begin{raggedright}
\begin{longtable*}{|p{4.2cm}|p{9cm}|p{1.8cm}|}
\hline
\textbf{Assessment Criterion} & \textbf{Descriptor} & \textbf{Marks} \\
\hline
\textbf{1. Identification and justification of five turning points} & 
Selects five relevant developments (conceptual, methodological, technological, institutional) and justifies their importance in shaping statistical thought. \newline
\textit{0–2:} Incomplete or poorly justified selection. \newline
\textit{3–4:} Reasonable choices, with limited justification. \newline
\textit{5:} Clear, well-motivated and historically grounded selection. & 
0–5 \\
\hline
\textbf{2. Explanation of interactions among ideas, tools, and institutions} & 
Demonstrates how intellectual, technological, and societal forces interacted to shape statistical philosophy. \newline
\textit{0–1:} Fragmented account with little synthesis. \newline
\textit{2–3:} Recognises key interactions but lacks depth or integration. \newline
\textit{4–5:} Coherent analysis of mutual reinforcement and historical context. & 
0–5 \\
\hline
\textbf{3. Engagement with epistemological concepts} & 
Articulates how statistical reasoning relates to ideas of uncertainty, inference, observation, and measurement. \newline
\textit{0–1:} Superficial or absent treatment. \newline
\textit{2–3:} Some conceptual reflection, but underdeveloped. \newline
\textit{4–5:} Strong engagement with epistemic foundations. & 
0–5 \\
\hline
\textbf{4. Use of assigned reading and historical material} & 
Integrates material from the chapter (e.g., Galileo, the printing press, van Leeuwenhoek, Laplace, etc.) to support the argument. \newline
\textit{0–1:} Little or no reference to the reading. \newline
\textit{2:} Uses examples but with minimal integration. \newline
\textit{3:} Demonstrates meaningful synthesis with the source material. & 
0–3 \\
\hline
\textbf{5. Coherence, structure, and originality} & 
Writing is well-organised and shows independent thought. Argument flows logically, with appropriate variation in style and pace. \newline
\textit{0:} Disorganised or difficult to follow. \newline
\textit{1:} Generally coherent, but uneven. \newline
\textit{2:} Clear and competent. \newline
\textit{3:} Persuasive, well-paced, and conceptually engaging. & 
0–3 \\
\hline
\multicolumn{2}{|r|}{\textbf{Total}} & \textbf{20} \\
\hline
\end{longtable*}
\end{raggedright}
\end{small}

\subsection{Question 4}\label{question-4}

Statistical reasoning begins with our wish to learn about something
large and often inaccessible by examining something smaller and
manageable. The credibility of this approach -- from observed data to
broader inference -- depends on how we conceptualise and structure the
relationship between what we observe and what we want to know.

This question asks that you examine the important terms and principles
that make this act of inference possible.

\textbf{Question:} What do statisticians mean by ``population'' and
``sample''? Define each term clearly, and explain the distinction
between them. How are they related in practice, and how does the method
of sampling affect the validity of estimates for population parameters
such as the mean and dispersion? Support your discussion with examples
where appropriate.

\textbf{{[}10 marks{]}}

\textbf{Answer}

\begin{small}
\begin{raggedright}
\begin{longtable*}{|p{4.2cm}|p{9cm}|p{1.8cm}|}
\hline
\textbf{Assessment Criterion} & \textbf{Descriptor} & \textbf{Marks} \\
\hline
\textbf{1. Definition and distinction: population vs. sample} & 
Provides clear, accurate definitions. Demonstrates understanding of how a sample is conceptually and inferentially linked to a population. \newline
\textit{0:} Definitions absent or incorrect. \newline
\textit{1:} Basic or vague explanation. \newline
\textit{2:} Clear, accurate, and well-articulated definitions and distinctions. & 
0–2 \\
\hline
\textbf{2. Relationship in practice} & 
Explains how samples are used to draw conclusions about populations; identifies the rationale for using samples. \newline
\textit{0:} No explanation or incorrect claim. \newline
\textit{1:} Partial understanding. \newline
\textit{2:} Correct and practically contextualised explanation. & 
0–2 \\
\hline
\textbf{3. Role of sampling method} & 
Identifies how sampling strategies (random, biased, etc.) influence the reliability of estimates like the mean and dispersion. \newline
\textit{0:} No discussion of sampling method. \newline
\textit{1:} Mentions method but lacks detail. \newline
\textit{2:} Analytically explains how sampling quality affects inferential accuracy. & 
0–2 \\
\hline
\textbf{4. Impact on estimates of population parameters} & 
Connects sampling quality to estimates of central tendency and variation. May address bias, variability, or representativeness. \newline
\textit{0:} No reference to estimation. \newline
\textit{1:} Simplistic account (e.g., just states “affects accuracy”). \newline
\textit{2:} Well-reasoned explanation with statistical relevance. & 
0–2 \\
\hline
\textbf{5. Use of relevant examples and clarity of expression} & 
Supports discussion with apt examples; communicates ideas clearly and logically. \newline
\textit{0:} Unclear or no examples. \newline
\textit{1:} Example provided but not integrated. \newline
\textit{2:} Well-chosen example(s) that enhance explanation. & 
0–2 \\
\hline
\multicolumn{2}{|r|}{\textbf{Total}} & \textbf{10} \\
\hline
\end{longtable*}
\end{raggedright}
\end{small}

\subsection{Question 5}\label{question-5}

Words shape our thoughts, and nowhere is this more consequential than in
science, where terminological precision goes hand-in-hand with
conceptual clarity. Statistical terms like ``random'' or ``stochastic''
carry specific meanings in the context of probabilistic logic and
mathematical formalism. Yet in everyday language, such terms are often
misused. They are flattened into colloquialisms that only hint at their
true meaning. This insidious slippage is more than semantic; it has
consequences for how we value knowledge.

Why does it matter if ``random'' is used imprecisely? How do scientific
concepts become confused, or even trivialised, when technical language
is absorbed into everyday language without regard for its analytic
structure?

\textbf{Question:} Discuss the scientific meaning of ``random'' and
contrast it with its colloquial usage. Why is this distinction important
for statistical reasoning, and how can imprecise language lead to
conceptual misunderstandings? In your answer, consider how terms like
``haphazard'' and ``unpredictable'' differ from ``random,'' and evaluate
the knowledge implications of using such words loosely in scientific or
public discourse.

\textbf{{[}10 marks{]}}

\textbf{Answer}

\begin{small}
\begin{raggedright}
\begin{longtable*}{|p{4.2cm}|p{9cm}|p{1.8cm}|}
\hline
\textbf{Assessment Criterion} & \textbf{Descriptor} & \textbf{Marks} \\
\hline
\textbf{1. Definition of scientific and colloquial meanings of “random”} & 
Provides a clear and accurate definition of “random” in statistical reasoning and contrasts it meaningfully with everyday usage. \newline
\textit{0:} Incorrect or missing definitions. \newline
\textit{1:} Partial or imprecise contrast. \newline
\textit{2:} Accurate definitions and well-articulated contrast. & 
0–2 \\
\hline
\textbf{2. Explanation of significance in statistical reasoning} & 
Explains why conceptual precision around “randomness” matters for designing, interpreting, or trusting statistical inference. \newline
\textit{0:} No justification or misunderstanding of significance. \newline
\textit{1:} Basic relevance noted but not developed. \newline
\textit{2:} Shows clear understanding of why terminological precision matters. & 
0–2 \\
\hline
\textbf{3. Discussion of related terms and conceptual confusion} & 
Evaluates how terms like “haphazard” or “unpredictable” differ from “random,” and discusses implications of terminological slippage. \newline
\textit{0:} No mention of related terms. \newline
\textit{1:} Terms mentioned but distinction not clearly drawn. \newline
\textit{2:} Analytically distinguishes and explores conceptual confusion. & 
0–2 \\
\hline
\textbf{4. Evaluation of consequences for knowledge or discourse} & 
Assesses how loose language affects scientific literacy or distorts public understanding. \newline
\textit{0:} No evaluation of broader consequences. \newline
\textit{1:} Mentions issue but lacks depth. \newline
\textit{2:} Engages thoughtfully with implications for knowledge/policy/discourse. & 
0–2 \\
\hline
\textbf{5. Clarity, expression, and structure} & 
Writing is coherent, conceptually organised, and shows linguistic control. \newline
\textit{0:} Disorganised or opaque. \newline
\textit{1:} Understandable but uneven. \newline
\textit{2:} Clear, persuasive, and well-structured. & 
0–2 \\
\hline
\multicolumn{2}{|r|}{\textbf{Total}} & \textbf{10} \\
\hline
\end{longtable*}
\end{raggedright}
\end{small}

\subsection{Question 6}\label{question-6}

Your task is to design a hypothetical study that could lead to a
statistical analysis using one of the following methods:

\begin{itemize}
\tightlist
\item
  One-way ANOVA
\item
  Simple linear regression
\item
  Pearson or Spearman correlation
\end{itemize}

Your study may involve field sampling, a laboratory experiment, or
observational data -- what matters is that your design aligns
meaningfully with the statistical method you choose.

In your answer, do the following:

\begin{enumerate}
\def\labelenumi{\arabic{enumi}.}
\tightlist
\item
  Describe your hypothetical experiment or sampling campaign.

  \begin{itemize}
  \tightlist
  \item
    Outline what you are investigating, how data will be collected, and
    what your variables are. Be clear about their measurement scale
    (categorical, continuous) and expected behaviour.
  \item
    Present this as a formally written Methods section suitable for a
    peer-review publication.
  \end{itemize}
\item
  Justify the statistical method you have chosen.

  \begin{itemize}
  \tightlist
  \item
    Explain why your design is appropriate for ANOVA, regression, or
    correlation.
  \end{itemize}
\item
  Formally state the null and alternative hypotheses as they would be
  tested in the chosen analysis.
\item
  Show a portion of the pseudo-data as one would see using the
  \texttt{head()} or \texttt{tail()} functions in R.

  \begin{itemize}
  \tightlist
  \item
    This should be a small, representative sample of the data you would
    collect.
  \end{itemize}
\item
  Describe the sequence of analytical steps you would take -- from raw
  data to final conclusion.

  \begin{itemize}
  \tightlist
  \item
    Include any relevant assumptions, diagnostic checks, or
    transformations that may be required before interpreting the
    results.
  \end{itemize}
\item
  Write a hypothetical Results section that summarises the findings of
  your analysis.

  \begin{itemize}
  \tightlist
  \item
    This should include a brief interpretation of the statistical
    output, including relevant pseudo-tables or pseudo-figures.
  \end{itemize}
\end{enumerate}

Your answer should reflect an understanding of the logic and structure
of statistical inference, from design to decision. You are welcome to
use R and RStudio to generate any data, tables, and graphs, should you
wish.

\textbf{{[}25 marks{]}}

\textbf{Answer}

\begin{small}
\begin{raggedright}
\begin{longtable*}{|p{4.2cm}|p{9cm}|p{1.8cm}|}
\hline
\textbf{Assessment Criterion} & \textbf{Descriptor} & \textbf{Marks} \\
\hline
\textbf{1. Experimental or sampling design (Methods section)} & 
Presents a hypothetical study with clear variables, measurement types (categorical/continuous), and logical data collection approach. Framed in the style of a peer-reviewed Methods section. \newline
\textit{0–2:} Vague, underdeveloped, or incoherent. \newline
\textit{3–4:} Adequate design, partially formalised. \newline
\textit{5:} Clear, plausible, and professionally structured. & 
0–5 \\
\hline
\textbf{2. Justification of chosen statistical method} & 
Explains why the method (ANOVA, regression, correlation) is appropriate based on the variables and study question. \newline
\textit{0–1:} Method chosen without justification. \newline
\textit{2–3:} Method mostly appropriate, with some justification. \newline
\textit{4:} Method fully justified and aligned with design. & 
0–4 \\
\hline
\textbf{3. Hypothesis formulation} & 
States null and alternative hypotheses as they would appear in formal statistical testing. Correctly aligned with method and data structure. \newline
\textit{0–1:} Incorrect or missing. \newline
\textit{2:} Present but informal or poorly structured. \newline
\textit{3:} Formally correct and clearly expressed. & 
0–3 \\
\hline
\textbf{4. Representative data sample} & 
Includes a small, clearly formatted table of (pseudo-)data to illustrate variables. May simulate `head()` or `tail()` output. \newline
\textit{0–1:} Absent or irrelevant data. \newline
\textit{2:} Present but unstructured or unclear. \newline
\textit{3:} Representative and appropriately formatted. & 
0–3 \\
\hline
\textbf{5. Analytical workflow (from raw data to inference)} & 
Describes logical steps: assumptions, transformations, model diagnostics, and inferential strategy. \newline
\textit{0–1:} Minimal or confused. \newline
\textit{2–3:} Partial sequence, some omissions. \newline
\textit{4:} Coherent, technically sound workflow. & 
0–4 \\
\hline
\textbf{6. Interpretation and Results summary} & 
Provides a hypothetical Results section interpreting the (pseudo-)statistical outcome, with mention of output tables/figures. \newline
\textit{0–1:} No interpretation or incoherent. \newline
\textit{2–3:} Interprets outcome, but superficially. \newline
\textit{4–5:} Thoughtful, succinct summary with clear output reference. & 
0–5 \\
\hline
\multicolumn{2}{|r|}{\textbf{Total}} & \textbf{25} \\
\hline
\end{longtable*}
\end{raggedright}
\end{small}

\textbf{TOTAL MARKS: 100}

\textbf{-- THE END --}




\end{document}
