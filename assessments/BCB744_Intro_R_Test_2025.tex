% Options for packages loaded elsewhere
\PassOptionsToPackage{unicode}{hyperref}
\PassOptionsToPackage{hyphens}{url}
\PassOptionsToPackage{dvipsnames,svgnames,x11names}{xcolor}
%
\documentclass[
  10t,
]{article}

\usepackage{amsmath,amssymb}
\usepackage{iftex}
\ifPDFTeX
  \usepackage[T1]{fontenc}
  \usepackage[utf8]{inputenc}
  \usepackage{textcomp} % provide euro and other symbols
\else % if luatex or xetex
  \usepackage{unicode-math}
  \defaultfontfeatures{Scale=MatchLowercase}
  \defaultfontfeatures[\rmfamily]{Ligatures=TeX,Scale=1}
\fi
\usepackage{lmodern}
\ifPDFTeX\else  
    % xetex/luatex font selection
\fi
% Use upquote if available, for straight quotes in verbatim environments
\IfFileExists{upquote.sty}{\usepackage{upquote}}{}
\IfFileExists{microtype.sty}{% use microtype if available
  \usepackage[]{microtype}
  \UseMicrotypeSet[protrusion]{basicmath} % disable protrusion for tt fonts
}{}
\makeatletter
\@ifundefined{KOMAClassName}{% if non-KOMA class
  \IfFileExists{parskip.sty}{%
    \usepackage{parskip}
  }{% else
    \setlength{\parindent}{0pt}
    \setlength{\parskip}{6pt plus 2pt minus 1pt}}
}{% if KOMA class
  \KOMAoptions{parskip=half}}
\makeatother
\usepackage{xcolor}
\setlength{\emergencystretch}{3em} % prevent overfull lines
\setcounter{secnumdepth}{-\maxdimen} % remove section numbering
% Make \paragraph and \subparagraph free-standing
\makeatletter
\ifx\paragraph\undefined\else
  \let\oldparagraph\paragraph
  \renewcommand{\paragraph}{
    \@ifstar
      \xxxParagraphStar
      \xxxParagraphNoStar
  }
  \newcommand{\xxxParagraphStar}[1]{\oldparagraph*{#1}\mbox{}}
  \newcommand{\xxxParagraphNoStar}[1]{\oldparagraph{#1}\mbox{}}
\fi
\ifx\subparagraph\undefined\else
  \let\oldsubparagraph\subparagraph
  \renewcommand{\subparagraph}{
    \@ifstar
      \xxxSubParagraphStar
      \xxxSubParagraphNoStar
  }
  \newcommand{\xxxSubParagraphStar}[1]{\oldsubparagraph*{#1}\mbox{}}
  \newcommand{\xxxSubParagraphNoStar}[1]{\oldsubparagraph{#1}\mbox{}}
\fi
\makeatother

\usepackage{color}
\usepackage{fancyvrb}
\newcommand{\VerbBar}{|}
\newcommand{\VERB}{\Verb[commandchars=\\\{\}]}
\DefineVerbatimEnvironment{Highlighting}{Verbatim}{commandchars=\\\{\}}
% Add ',fontsize=\small' for more characters per line
\usepackage{framed}
\definecolor{shadecolor}{RGB}{248,248,248}
\newenvironment{Shaded}{\begin{snugshade}}{\end{snugshade}}
\newcommand{\AlertTok}[1]{\textcolor[rgb]{0.94,0.16,0.16}{#1}}
\newcommand{\AnnotationTok}[1]{\textcolor[rgb]{0.56,0.35,0.01}{\textbf{\textit{#1}}}}
\newcommand{\AttributeTok}[1]{\textcolor[rgb]{0.13,0.29,0.53}{#1}}
\newcommand{\BaseNTok}[1]{\textcolor[rgb]{0.00,0.00,0.81}{#1}}
\newcommand{\BuiltInTok}[1]{#1}
\newcommand{\CharTok}[1]{\textcolor[rgb]{0.31,0.60,0.02}{#1}}
\newcommand{\CommentTok}[1]{\textcolor[rgb]{0.56,0.35,0.01}{\textit{#1}}}
\newcommand{\CommentVarTok}[1]{\textcolor[rgb]{0.56,0.35,0.01}{\textbf{\textit{#1}}}}
\newcommand{\ConstantTok}[1]{\textcolor[rgb]{0.56,0.35,0.01}{#1}}
\newcommand{\ControlFlowTok}[1]{\textcolor[rgb]{0.13,0.29,0.53}{\textbf{#1}}}
\newcommand{\DataTypeTok}[1]{\textcolor[rgb]{0.13,0.29,0.53}{#1}}
\newcommand{\DecValTok}[1]{\textcolor[rgb]{0.00,0.00,0.81}{#1}}
\newcommand{\DocumentationTok}[1]{\textcolor[rgb]{0.56,0.35,0.01}{\textbf{\textit{#1}}}}
\newcommand{\ErrorTok}[1]{\textcolor[rgb]{0.64,0.00,0.00}{\textbf{#1}}}
\newcommand{\ExtensionTok}[1]{#1}
\newcommand{\FloatTok}[1]{\textcolor[rgb]{0.00,0.00,0.81}{#1}}
\newcommand{\FunctionTok}[1]{\textcolor[rgb]{0.13,0.29,0.53}{\textbf{#1}}}
\newcommand{\ImportTok}[1]{#1}
\newcommand{\InformationTok}[1]{\textcolor[rgb]{0.56,0.35,0.01}{\textbf{\textit{#1}}}}
\newcommand{\KeywordTok}[1]{\textcolor[rgb]{0.13,0.29,0.53}{\textbf{#1}}}
\newcommand{\NormalTok}[1]{#1}
\newcommand{\OperatorTok}[1]{\textcolor[rgb]{0.81,0.36,0.00}{\textbf{#1}}}
\newcommand{\OtherTok}[1]{\textcolor[rgb]{0.56,0.35,0.01}{#1}}
\newcommand{\PreprocessorTok}[1]{\textcolor[rgb]{0.56,0.35,0.01}{\textit{#1}}}
\newcommand{\RegionMarkerTok}[1]{#1}
\newcommand{\SpecialCharTok}[1]{\textcolor[rgb]{0.81,0.36,0.00}{\textbf{#1}}}
\newcommand{\SpecialStringTok}[1]{\textcolor[rgb]{0.31,0.60,0.02}{#1}}
\newcommand{\StringTok}[1]{\textcolor[rgb]{0.31,0.60,0.02}{#1}}
\newcommand{\VariableTok}[1]{\textcolor[rgb]{0.00,0.00,0.00}{#1}}
\newcommand{\VerbatimStringTok}[1]{\textcolor[rgb]{0.31,0.60,0.02}{#1}}
\newcommand{\WarningTok}[1]{\textcolor[rgb]{0.56,0.35,0.01}{\textbf{\textit{#1}}}}

\providecommand{\tightlist}{%
  \setlength{\itemsep}{0pt}\setlength{\parskip}{0pt}}\usepackage{longtable,booktabs,array}
\usepackage{calc} % for calculating minipage widths
% Correct order of tables after \paragraph or \subparagraph
\usepackage{etoolbox}
\makeatletter
\patchcmd\longtable{\par}{\if@noskipsec\mbox{}\fi\par}{}{}
\makeatother
% Allow footnotes in longtable head/foot
\IfFileExists{footnotehyper.sty}{\usepackage{footnotehyper}}{\usepackage{footnote}}
\makesavenoteenv{longtable}
\usepackage{graphicx}
\makeatletter
\newsavebox\pandoc@box
\newcommand*\pandocbounded[1]{% scales image to fit in text height/width
  \sbox\pandoc@box{#1}%
  \Gscale@div\@tempa{\textheight}{\dimexpr\ht\pandoc@box+\dp\pandoc@box\relax}%
  \Gscale@div\@tempb{\linewidth}{\wd\pandoc@box}%
  \ifdim\@tempb\p@<\@tempa\p@\let\@tempa\@tempb\fi% select the smaller of both
  \ifdim\@tempa\p@<\p@\scalebox{\@tempa}{\usebox\pandoc@box}%
  \else\usebox{\pandoc@box}%
  \fi%
}
% Set default figure placement to htbp
\def\fps@figure{htbp}
\makeatother

% preamble.tex

% --- Document Structure and Layout ---

\usepackage[a4paper, total={6in, 8in}]{geometry}

% --- Paragraph settings ---

\setlength{\parindent}{0pt}
\setlength{\parskip}{6pt}

% --- Color Definitions ---

\usepackage[x11names]{xcolor} % Required for specifying custom colors, load before tcolorbox
\definecolor{headingblue}{RGB}{23,48,191}
\definecolor{boxtitle}{HTML}{F0F4F8}
\definecolor{boxbody}{HTML}{FBFDFF}
\definecolor{mainboxframe}{HTML}{F0F4F8}
\definecolor{subboxframe}{HTML}{F0F4F8}
\definecolor{crimson}{HTML}{880000}
\definecolor{monocolor}{RGB}{64,224,208}

% --- Fonts and Encoding ---

\usepackage{fontspec}         % Allows font specification
% \usepackage{amsmath}          % For math symbols

% Main Font
\setmainfont[
  UprightFont = *-Regular,
  ItalicFont = *-Italic,
  ItalicFeatures = { SmallCapsFont = *-Italic },
  SlantedFont = *-Regular,
  SlantedFeatures= { FakeSlant=0.2 },
  BoldFont = *-Bold,
  BoldFeatures = { SmallCapsFont = *-Bold },
  BoldItalicFont = *-BoldItalic,
  BoldItalicFeatures = { SmallCapsFont = *-BoldItalic },
  BoldSlantedFont= *-Bold,
  BoldSlantedFeatures= { FakeSlant=0.2, SmallCapsFont = *-Bold },
  SmallCapsFont = *-Regular,
  SmallCapsFeatures={ RawFeature=+smcp },
  Ligatures=TeX,
  Numbers={OldStyle, Proportional}
]{StixTwoText}

% Math Font
\setmathfont{StixTwoMath.otf}

% Monospace Font
\setmonofont[
  Scale=0.84
]{FiraCode Nerd Font}
\renewcommand{\ttfamily}{\small\fontspec{FiraCode Nerd Font}\color{DeepSkyBlue4}}
\renewcommand{\texttt}[1]{{\ttfamily #1}}

% --- Packages for Tables ---

\usepackage{array}            % For table column width specification
\usepackage{booktabs}         % For table rules
\usepackage{ragged2e}         % For text alignment (used with \newcolumntype)

% --- Headers and Footers ---

\usepackage{fancyhdr}
\pagestyle{fancy}
\renewcommand{\sectionmark}[1]{\markright{#1}{}}
\fancyhf{}
\fancyhead[LE,RO]{\thepage}
\fancyhead[LO]{\textsc{\MakeLowercase{\leftmark}}}
\fancyhead[RE]{\textsc{\MakeLowercase{\rightmark}}}

% --- Other Packages ---

\usepackage[version=4]{mhchem}% Formula subscripts using \ce{}

% --- Key Terms

\newcommand{\keyterm}[1]{\textsc{#1}}

%% Create a command for color emphasis
\newcommand{\highlight}[1]{\textcolor{crimson}{#1}}

% --- Boxes ---

% Define the mdframed environment
\usepackage{float}
\usepackage{mdframed}

% 1) Define a new float environment called "boxfloat"
\newfloat{boxfloat}{htbp}{lob}
\floatname{boxfloat}{Box}

% 3) Define the environment that wraps mdframed in a float
\newenvironment{boxedfloat}[2][]{%
  % Advance the box counter to produce "Chapter.BoxNo"
  \refstepcounter{boxcounter}%
  % Begin the float environment
  \begin{boxfloat}[htbp]
  % Begin the mdframed styling
  \begin{mdframed}[
    backgroundcolor=gray!5,
    innertopmargin=6pt,
    innerbottommargin=6pt,
    innerrightmargin=6pt,
    innerleftmargin=6pt,
    linewidth=0.25pt,
    linecolor=black,
    roundcorner=8pt,    % or 0pt if you prefer sharp corners
    skipabove=12pt,     % vertical space above the box
    skipbelow=12pt,     % vertical space below the box
    innermargin=0pt,
    outermargin=0pt
  ]%
    % Typeset the box heading: "Box 1.1. My Title"
    \setlength{\parindent}{0em}%
    \setlength{\parskip}{3pt}%
    \RaggedRight
    % Both "Box" and the user-supplied title are in small caps
    \small% switch the box contents to smaller text
    {\scshape Box \theboxcounter. #2}\par
    \vspace{6pt} % a little space after the heading
}{%
    \end{mdframed}
    \end{boxfloat}
}

% --- sansblock Environment ---

\usepackage{sourcesanspro}    % Load Source Sans Pro
\setsansfont{Source Sans Pro} % Set it as the sans-serif font
\newenvironment{sansblock}[1]
    {\small\sffamily\raggedright{\scshape #1}\ } % Ensure small caps for the title
  {} % End environment: no special commands needed

% --- Custom Column Type (using ragged2e) ---

\newcolumntype{R}[1]{>{\RaggedRight}p{#1}}

% ---  Margin Notes ---

\usepackage{marginnote}
\renewcommand*{\marginfont}{\footnotesize\itshape} % Style for margin notes

%% Set margin note outer margin to 0.7in
\setlength{\marginparwidth}{1.25in}

% --- Epigraph ---

\usepackage{epigraph}
\setlength\epigraphwidth{.9\textwidth}
\newenvironment{quotepara}
  {\itshape\raggedright\small\setlength{\parskip}{0.5em}} % Add small space between paragraphs
  {}
\renewcommand{\textflush}{quotepara}

%% Define a new epigraph environment without the horizontal rule and source/author
\newenvironment{simpleepigraph}
  {\begin{list}{}%
      {\setlength{\leftmargin}{2em}% Left margin
       \setlength{\rightmargin}{2em}% Right margin
       \setlength{\topsep}{1em}% Space above the epigraph
       \setlength{\itemsep}{0pt}% Space between items (irrelevant here)
       \setlength{\parsep}{0pt}}% Space between paragraphs
   \item\relax\raggedright\small} % Apply ragged-right and italic style for the epigraph text
  {\end{list}}

% --- Small Caps ---

\newcommand{\flatcaps}[1]{\textsc{\MakeLowercase{#1}}}

% --- Lists ---

%% General settings for all lists
\usepackage{enumitem}

% Global settings following Bringhurst's principles
% A global default to keep lists tight, but still allow subtle spacing:
\setlist{
  nosep,         % No extra space between items
  topsep=0.6em,  % A bit of space before/after the list
  parsep=0pt,
  partopsep=0pt
}

% First-level itemize (unordered) lists:
\setlist[itemize,1]{
  label=\textbullet,
  labelsep=0.4em,        % Space from bullet to text
  labelwidth=1em,        % Horizontal space set aside for bullet
  leftmargin=\dimexpr 1em + 0.4em\relax,
  itemindent=0pt,
  listparindent=0pt,
  align=left
}

% Second-level itemize, with a subtler symbol:
\setlist[itemize,2]{
  label=--,
  labelsep=0.4em,
  labelwidth=1em,
  leftmargin=\dimexpr 1em + 0.4em\relax,
  itemindent=0pt,
  listparindent=0pt,
  align=left
}

% First-level enumerate (ordered) lists:
\setlist[enumerate,1]{
  label=\arabic*.,
  labelsep=0.4em,
  labelwidth=1em,
  leftmargin=\dimexpr 1em + 0.4em\relax,
  itemindent=0pt,
  listparindent=0pt,
  align=left
}

% Second-level enumerate (letters, or you could do roman numerals):
\setlist[enumerate,2]{
  label=\alph*.,
  labelsep=0.4em,
  labelwidth=1em,
  leftmargin=\dimexpr 1em + 0.4em\relax,
  itemindent=0pt,
  listparindent=0pt,
  align=left
}

% --- Custom Chapter/Section Styles ---

\usepackage[compact]{titlesec} % Allows creating custom chapter styles
\titleformat{\chapter}[display]
  {\fontsize{60}{62}\bfseries}
  {\thechapter}
  {0pt}
  {\huge\noindent}
\titlespacing*{\chapter}{0pt}{0pt}{40pt}

\titleformat{\section}
  {\normalsize\normalfont}
  {\thesection}
  {0.6em}
  {\flatcaps}
\titlespacing*{\section}{0pt}{1\baselineskip}{1\baselineskip}

\titleformat{\subsection}[block]
  {\normalsize\normalfont} % defines the font size and style for the entire subsection heading, including both the number and the title
  {\thesubsection} % defines the format of the subsection number
  {1em} % the horizontal space between the subsection number and the title
  {\itshape} % defines the format of the subsection title itself
\titlespacing*{\subsection}{0pt}{1\baselineskip}{1\baselineskip}

\titleformat{\subsubsection}[runin]
  {\normalsize\normalfont} % defines the font size and style for the entire subsection heading, including both the number and the title
  {\thesubsubsection} % defines the format of the subsection number
  {1em} % the horizontal space between the subsection number and the title
  {\itshape}[.] % defines the format of the subsection title itself
\titlespacing*{\subsubsection}{0pt}{1\baselineskip}{1\baselineskip}

\titleformat{\paragraph}[runin]
  {\flatcaps}
  {\theparagraph}
  {0pt}
  {}

% --- Footnotes ---

\usepackage[norule,ragged,hang]{footmisc}  % Load footmisc with ragged option
\renewcommand{\footnotelayout}{\RaggedRight\footnotesize} % Typeset footnotes in \RaggedRight
\setlength{\footnotemargin}{1.5em}    % Adjust space between number and text
\makeatletter
\renewcommand{\@makefntext}[1]{%
    \parindent 1em%                    Set parindent for footnote text
    \noindent
    \hb@xt@ 1.8em{%                   Set hanging indent for footnote text
        \hss\@thefnmark.%
    }
    \RaggedRight #1%                 Typeset footnote text ragged right
}
\makeatother

% --- Captions ---

\usepackage{caption}
\captionsetup{
  font={small},
  labelfont={bf},
  textfont={},
  width=0.9\textwidth,
  justification=justified,
  labelformat=default,
  labelsep=period,
  format=plain
}
\renewcommand{\captionlabelfont}{\bfseries\scshape}

% --- Hyperlinks ---

\usepackage{hyperref}           % Load after most other packages, but before cleveref
\hypersetup{
    colorlinks=true,
    linkcolor=blue,
    filecolor=magenta,
    urlcolor=cyan,
    pdftitle={Overleaf Example},
    pdfpagemode=FullScreen,
    }

\urlstyle{same}

% --- Miscellaneous ---

%% Define a new command to print the current page number to the console
\ifluatex
  \usepackage{luacode}
  \usepackage{shellesc}
  \newcommand{\printpagenumber}{%
    \directlua{
      local pagenumber = tex.count.page
      print(string.format("Currently processing page: %d", pagenumber))
    }
  }
\fi

\usepackage{etoolbox}          % General package for patching commands
\usepackage{iftex}             % Detects the engine used
\usepackage{ellipsis}          % Fixes spacing around ellipses
\AddToHook{env/Highlighting/begin}{\small} % Set the code chunk font size globally

%% Use lining fonts
\newcommand\lining{\addfontfeatures{Numbers={Monospaced, Lining}}}
\AtBeginEnvironment{tabular}{\lining} % In tables
\renewcommand{\theequation}{ {\lining\arabic{equation}}} % For equation numbers

% --- Index (if needed) ---

\usepackage{makeidx}
\makeindex

% --- Other Typography Settings ---

\frenchspacing                % Single space after periods
\tolerance=400                % Default is 200; higher values allow more relaxed line-breaking.
\emergencystretch=3em         % Adds additional space to help line-breaking.
\hyphenpenalty=20             % Default is 50; lower values encourage hyphenation.

% --- Microtype Settings (adjust only if needed) ---

\usepackage{microtype}        % Improves typography
\microtypesetup{
   tracking = true,
   protrusion=true,
   expansion=true,
   factor = 1100,
   stretch = 15,
   shrink = 15
}

\makeatletter
\@ifpackageloaded{caption}{}{\usepackage{caption}}
\AtBeginDocument{%
\ifdefined\contentsname
  \renewcommand*\contentsname{Table of contents}
\else
  \newcommand\contentsname{Table of contents}
\fi
\ifdefined\listfigurename
  \renewcommand*\listfigurename{List of Figures}
\else
  \newcommand\listfigurename{List of Figures}
\fi
\ifdefined\listtablename
  \renewcommand*\listtablename{List of Tables}
\else
  \newcommand\listtablename{List of Tables}
\fi
\ifdefined\figurename
  \renewcommand*\figurename{Figure}
\else
  \newcommand\figurename{Figure}
\fi
\ifdefined\tablename
  \renewcommand*\tablename{Table}
\else
  \newcommand\tablename{Table}
\fi
}
\@ifpackageloaded{float}{}{\usepackage{float}}
\floatstyle{ruled}
\@ifundefined{c@chapter}{\newfloat{codelisting}{h}{lop}}{\newfloat{codelisting}{h}{lop}[chapter]}
\floatname{codelisting}{Listing}
\newcommand*\listoflistings{\listof{codelisting}{List of Listings}}
\makeatother
\makeatletter
\makeatother
\makeatletter
\@ifpackageloaded{caption}{}{\usepackage{caption}}
\@ifpackageloaded{subcaption}{}{\usepackage{subcaption}}
\makeatother
\makeatletter
\@ifpackageloaded{sidenotes}{}{\usepackage{sidenotes}}
\@ifpackageloaded{marginnote}{}{\usepackage{marginnote}}
\makeatother

\usepackage{bookmark}

\IfFileExists{xurl.sty}{\usepackage{xurl}}{} % add URL line breaks if available
\urlstyle{same} % disable monospaced font for URLs
\hypersetup{
  pdftitle={BCB744: Intro R Test},
  pdfauthor={Smit, A. J.},
  colorlinks=true,
  linkcolor={blue},
  filecolor={blue},
  citecolor={blue},
  urlcolor={blue},
  pdfcreator={LaTeX via pandoc}}


\title{BCB744: Intro R Test}
\author{Smit, A. J.}
\date{2025-03-17}

\begin{document}
\maketitle


\section{About the test}\label{about-the-test}

The Intro R Test will start at 8:30 on 17 March, 2025 and you have until
08:30 on 18 March to complete it. The Theory Test must be conducted on
campus, and the Practical Test at home or anywhere you are comfortable
working. The test constitutes a key component of Continuous Assessment
(CA) and are designed to prepare you for the final exam.

The test consists of two parts:

\subsection{Theory Test (30\%)}\label{theory-test-30}

This is a written, closed-book assessment where you will be tested on
theoretical concepts. The only resource available during this test is
RStudio, the R help system, your memory, and your mind.

\subsection{Practical Test (70\%)}\label{practical-test-70}

In this open-book coding assessment, you will apply your theoretical
knowledge to real data problems. While you may reference online
materials (including ChatGPT), collaboration with peers is strictly
prohibited.

\section{Assessment Policy}\label{assessment-policy}

{The marks indicated for each section reflect the relative weight (and
hence depth expected in your response) rather than a rigid checklist of
individual points.} Your answer should demonstrate a comprehensive
understanding of the concepts and techniques required, showing
thoughtful integration of multiple R skills. Higher marks will be
awarded for solutions that demonstrate not only technical correctness
but also elegant code, insightful analysis, and clear communication of
findings. We are assessing your ability to think systematically through
complex data problems, make appropriate methodological choices, and
present your findings in a coherent narrative that reveals meaningful
patterns in the data. Your code should be well-structured, adequately
commented, and reflect good programming practices.

Please refer to the
\href{https://tangledbank.netlify.app/BCB744/BCB744_index.html\#sec-policy}{Assessment
Policy} for more information on the test format and rules.

\section{Theory Test}\label{theory-test}

{\textbf{This is the closed book assessment.}}

Below is a set of questions to answer. You must answer all questions in
the allocated time of 3-hr. Please write your answers in a neatly
formatted Word document and submit it to the iKamva platform.

Clearly indicate the question number and provide detailed explanations
for your answers. Use Word's headings and subheadings facility to
structure your document logically.

Naming convention: \texttt{Intro\_R\_Test\_Theory\_YourSurname.docx}

\subsection{Question 1}\label{question-1}

You are a research assistant who have just been given your first job.
You are asked to analyse a dataset about patterns of extreme heat in the
ocean and the possible role that ocean currents (specifically, eddies)
might play in modulating the patterns of extreme sea surface temperature
extremes in space and time.

Being naive and relatively inexperienced, and misguided by your
exaggerated sense of preparedness as young people tend to do, you gladly
accept the task and start by exploring the data. You notice that the
dataset is quite large, and you have no idea what's happening, what you
are doing, why you are doing it, or what you are looking for. Ten
minutes into the job you start to question your life choices. Your
feeling of bewilderment is compounded by the fact that, when you examine
the data (the output of the \texttt{head()} and \texttt{tail()} commands
is shown below), the entries seem confusing.

\begin{Shaded}
\begin{Highlighting}[]
\NormalTok{fpath }\OtherTok{\textless{}{-}} \StringTok{"/Volumes/OceanData/spatial/processed/WBC/misc\_results"}
\NormalTok{fname }\OtherTok{\textless{}{-}} \StringTok{"KC{-}MCA{-}data{-}2013{-}01{-}01{-}2022{-}12{-}31{-}bbox{-}v1\_ma\_14day\_detrended.csv"}
\NormalTok{data }\OtherTok{\textless{}{-}} \FunctionTok{read.csv}\NormalTok{(}\FunctionTok{file.path}\NormalTok{(fpath, fname))}
\end{Highlighting}
\end{Shaded}

\begin{Shaded}
\begin{Highlighting}[]
\SpecialCharTok{\textgreater{}} \FunctionTok{nrow}\NormalTok{(data)}
\NormalTok{[}\DecValTok{1}\NormalTok{] }\DecValTok{53253434}

\SpecialCharTok{\textgreater{}} \FunctionTok{head}\NormalTok{(data)}
\NormalTok{           t     lon    lat      ex    ke}
\DecValTok{1} \DecValTok{2013{-}01{-}01} \FloatTok{121.875} \FloatTok{34.625} \SpecialCharTok{{-}}\FloatTok{0.7141} \FloatTok{2e{-}04}
\DecValTok{2} \DecValTok{2013{-}01{-}01} \FloatTok{121.875} \FloatTok{34.625} \SpecialCharTok{{-}}\FloatTok{0.8027} \FloatTok{2e{-}04}
\DecValTok{3} \DecValTok{2013{-}01{-}02} \FloatTok{121.875} \FloatTok{34.625} \SpecialCharTok{{-}}\FloatTok{0.8916} \FloatTok{2e{-}04}
\DecValTok{4} \DecValTok{2013{-}01{-}02} \FloatTok{121.875} \FloatTok{34.625} \SpecialCharTok{{-}}\FloatTok{0.9751} \FloatTok{2e{-}04}
\DecValTok{5} \DecValTok{2013{-}01{-}03} \FloatTok{121.875} \FloatTok{34.625} \SpecialCharTok{{-}}\FloatTok{1.0589} \FloatTok{3e{-}04}
\DecValTok{6} \DecValTok{2013{-}01{-}03} \FloatTok{121.875} \FloatTok{34.625} \SpecialCharTok{{-}}\FloatTok{1.1406} \FloatTok{3e{-}04}

\SpecialCharTok{\textgreater{}} \FunctionTok{tail}\NormalTok{(data)}
\NormalTok{                  t     lon    lat     ex      ke}
\DecValTok{53253429} \DecValTok{2022{-}12{-}29} \FloatTok{174.375} \FloatTok{44.875} \FloatTok{0.4742} \SpecialCharTok{{-}}\FloatTok{0.0049}
\DecValTok{53253430} \DecValTok{2022{-}12{-}29} \FloatTok{174.375} \FloatTok{44.875} \FloatTok{0.4856} \SpecialCharTok{{-}}\FloatTok{0.0049}
\DecValTok{53253431} \DecValTok{2022{-}12{-}30} \FloatTok{174.375} \FloatTok{44.875} \FloatTok{0.4969} \SpecialCharTok{{-}}\FloatTok{0.0050}
\DecValTok{53253432} \DecValTok{2022{-}12{-}30} \FloatTok{174.375} \FloatTok{44.875} \FloatTok{0.5169} \SpecialCharTok{{-}}\FloatTok{0.0050}
\DecValTok{53253433} \DecValTok{2022{-}12{-}31} \FloatTok{174.375} \FloatTok{44.875} \FloatTok{0.5367} \SpecialCharTok{{-}}\FloatTok{0.0051}
\DecValTok{53253434} \DecValTok{2022{-}12{-}31} \FloatTok{174.375} \FloatTok{44.875} \FloatTok{0.5465} \SpecialCharTok{{-}}\FloatTok{0.0051}
\end{Highlighting}
\end{Shaded}

You resign yourself to admitting that you don't understand much, but at
the risk of sounding like a fool when you go to your professor, you
decide to do as much of the preparation you can do so that you at least
have something to show for your time.

\begin{enumerate}
\def\labelenumi{\alph{enumi}.}
\tightlist
\item
  What will you take back to your professor to show that you have
  prepared yourself as fully as possible? For example:

  \begin{itemize}
  \tightlist
  \item
    What is in your ability to understand about the study and the nature
    of the data?
  \item
    What will you do for yourself to better understand the task at hand?
  \item
    What do you understand about the data?
  \item
    What will you do to aid your understanding of the data?
  \item
    What will your next steps be going forward?
  \end{itemize}
\item
  What will you need from your professor to help you understand the data
  and the task at hand so that you are well equipped to tackle the
  problem?
\end{enumerate}

\textbf{{[}15 marks{]}}

\textbf{Answer}

\begin{itemize}
\tightlist
\item
  I am able to understand what the concept of `extreme heat' is, and
  what ocean eddies are -- all I need to do is find some papers about it
  and do broad reading around these concepts. So, I will start by
  reading up on these concepts.
\item
  I can see from the columns that there appears to be three independent
  variables (\texttt{lon}, \texttt{lat}, and \texttt{t}) and two
  dependent variables (\texttt{ex} and \texttt{ke}). I will need to
  understand what these variables are, and how they relate to each
  other. It is easy to see that \texttt{lon} and \texttt{lat} are the
  longitude and latitude of the data points, and that \texttt{t} is the
  date of the data point. I will need to understand what the \texttt{ex}
  and \texttt{ke} variables are, and how they relate to the \texttt{lon}
  and \texttt{lat} variables. Presumably \texttt{ex} and \texttt{ke} are
  the extreme heat and ocean eddies, respectively. I'll confirm with the
  professor.
\item
  Because I have \texttt{lon} and \texttt{lat}, I can make a map of the
  study area. By making a map of the study area for one or a few days in
  the dataset, I can get a sense of the spatial distribution of the
  data. I can also plot the \texttt{ex} and \texttt{ke} data to see what
  the data look like. Because the data cover the period 2013-2022, I
  know that I can create a map for each day (a time-series analysis
  might eventually be needed?), and that is probably where the analysis
  will takle me later once I have confirmed my thinking with the
  professor. If I am really proactive and want to seriously impress the
  professor, I'll make an animation of the data to show the temporal
  evolution of revealed patterns in the data over time. This will
  clearly show the processes operating there. A REALLY informed mind
  will be able to even go as far as understanding what the analysis
  shoud entail, but, admittedly, this will require a deep subject matter
  understanding, which you might not possess at the moment, but which is
  nevertheless not beyond your reach to attain without guidance.
\item
  I can conclude that the data reveal some dynamical process (I infer
  `dynamical' from the fact that we have time-series data, and
  time-series reveal dynamics).
\item
  Knowing what the geographical region is from the map I created and
  what is happening there that might be of interest to the study, I can
  make some guesses about what the analysis will be.
\item
  FYI, what basic reseach would reveal include the following (not for
  marks):

  \begin{itemize}
  \tightlist
  \item
    you'd see that it is an ocean region south of South Africa;
  \item
    once you know the region covered, you can read about the processes
    operating in the region that the data cover;
  \item
    because the temperature spatially defines the Agulhas Current, you
    can infer that the study is about the Agulhas Current
  \item
    plotting \texttt{ke} will reveal eddies in the Agulhas Current;
  \item
    you can read about the Agulhas Current and its eddies and think
    about how eddies might affect the temperature in the region -- both
    of these are dynamical processes.
  \end{itemize}
\item
  I will need to understand what the data are telling me, and what the
  variables mean. I will need to understand what the \texttt{ex} and
  \texttt{ke} variables are, and how they relate to the \texttt{lon} and
  \texttt{lat} variables.
\item
  Having discovered all these things simply by doing a basic first-stab
  analyses, I can prepare a report of my cursory findings and draw of a
  list of things I know, toghether with suggested further avenues for
  exploration. I will take this to the professor to confirm my
  understanding and to get guidance on how to proceed.
\item
  I will also add a list of the things I cannot know from the data, and
  what I need to know from the professor to proceed.
\item
  There is also something strange happening with the data. It seems that
  there are duplicate data entries (two occurrences of each combination
  of \texttt{lat} x \texttt{lon} x \texttt{t} resulting in duplicated
  values for each spatio-temporal point of \texttt{ke} and a pair of
  dissimilar values for \texttt{ex}). I will need to understand why this
  is the case. Clearly this is incorrect, and this points to
  pre-processing errors somewhere. I will have to ask the professor to
  give me access to all pro-processing scripts and the raw data to see
  if I can trace the error back to its source.
\item
  If I was this professor, I'd be immensepy mpressed by tyour proactive
  approach to the problem. You are showing that you are not just a
  passive learner, but that you are actively engaging with the data and
  the problem at hand. This is a very good sign of a good researcher in
  the making. In my mind, I'd seriously think about finding you a salary
  for permanent employment in my lab.
\end{itemize}

\subsection{Question 2}\label{question-2}

Please translate the following code into English by providing an
explanation for each line:

\begin{Shaded}
\begin{Highlighting}[]
\NormalTok{monthlyData }\OtherTok{\textless{}{-}}\NormalTok{ dailyData }\SpecialCharTok{\%\textgreater{}\%}
\NormalTok{    dplyr}\SpecialCharTok{::}\FunctionTok{mutate}\NormalTok{(}\AttributeTok{t =} \FunctionTok{asPOSIXct}\NormalTok{(t)) }\SpecialCharTok{\%\textgreater{}\%}
\NormalTok{    dplyr}\SpecialCharTok{::}\FunctionTok{mutate}\NormalTok{(}\AttributeTok{month =} \FunctionTok{floor\_date}\NormalTok{(t, }\AttributeTok{unit =} \StringTok{"month"}\NormalTok{)) }\SpecialCharTok{\%\textgreater{}\%}
\NormalTok{    dplyr}\SpecialCharTok{::}\FunctionTok{group\_by}\NormalTok{(lon, lat, month) }\SpecialCharTok{\%\textgreater{}\%}
\NormalTok{    dplyr}\SpecialCharTok{::}\FunctionTok{summarise}\NormalTok{(}\AttributeTok{temp =} \FunctionTok{mean}\NormalTok{(temp, }\AttributeTok{na.rm =} \ConstantTok{TRUE}\NormalTok{)) }\SpecialCharTok{\%\textgreater{}\%}
\NormalTok{    dplyr}\SpecialCharTok{::}\FunctionTok{mutate}\NormalTok{(}\AttributeTok{year =} \FunctionTok{year}\NormalTok{(month)) }\SpecialCharTok{\%\textgreater{}\%}
\NormalTok{    dplyr}\SpecialCharTok{::}\FunctionTok{group\_by}\NormalTok{(lon, lat) }\SpecialCharTok{\%\textgreater{}\%}
\NormalTok{    dplyr}\SpecialCharTok{::}\FunctionTok{mutate}\NormalTok{(}\AttributeTok{num =} \FunctionTok{seq}\NormalTok{(}\DecValTok{1}\SpecialCharTok{:}\FunctionTok{length}\NormalTok{(temp))) }\SpecialCharTok{\%\textgreater{}\%}
\NormalTok{    dplyr}\SpecialCharTok{::}\FunctionTok{ungroup}\NormalTok{()}
\end{Highlighting}
\end{Shaded}

In your answer, simply refer to the line numbers (1-9) before each line
of code and provide an explanation for each line.

\textbf{{[}10 marks{]}}

\textbf{Answer}

\begin{itemize}
\tightlist
\item
  Line 1: The variable \texttt{monthlyData} is created by starting with
  \texttt{dailyData}, which is a dataset containing daily records.
\item
  Line 2: The \texttt{mutate()} function is used to convert the column
  \texttt{t} (presumably a date or timestamp) into a POSIXct datetime
  format. This ensures that \texttt{t} is stored in a standardised
  date-time format suitable for time-based operations.
\item
  Line 3: The \texttt{mutate()} function is again used to create a new
  column \texttt{month}, which is derived from \texttt{t}. The
  \texttt{floor\_date()} function rounds down the date to the first day
  of the corresponding month, effectively extracting the month from
  \texttt{t}.
\item
  Line 4: The \texttt{group\_by()} function groups the dataset by
  \texttt{lon} (longitude), \texttt{lat} (latitude), and \texttt{month}.
  This means subsequent operations will be performed separately for each
  unique combination of these three variables.
\item
  Line 5: The \texttt{summarise()} function computes the mean
  temperature (\texttt{temp}) for each group. The
  \texttt{na.rm\ =\ TRUE} argument ensures that missing values
  (\texttt{NA}) are ignored in the calculation.
\item
  Line 6: The \texttt{mutate()} function creates a new column,
  \texttt{year}, extracting the year from the \texttt{month} column.
  This provides an explicit reference to the year of each data entry.
\item
  Line 7: The \texttt{group\_by()} function is applied again, but this
  time only by \texttt{lon} and \texttt{lat}. This modifies the grouping
  structure to remove the month grouping while retaining spatial
  grouping.
\item
  Line 8: \texttt{The\ mutate()} function adds a new column,
  \texttt{num}, which assigns a sequence of numbers
  (\texttt{1:length(temp)}) to the grouped data. This effectively
  creates an index for each record within each longitude-latitude group.
\item
  Line 9: The \texttt{ungroup()} function removes all grouping, ensuring
  that further operations on \texttt{monthlyData} are performed on the
  entire dataset rather than within groups.
\end{itemize}

\subsection{Question 3}\label{question-3}

What is `Occam's Razor'?

\textbf{{[}5 marks{]}}

\textbf{Answer}

Occam's Razor is sometimes attributed to the 14th-century philosopher
William of Ockham, is a principle of parsimony that states: ``Entities
should not be multiplied beyond necessity.'' It is relevant to the
BCB744 module because the principle of Occam's Razor is often
interpreted as ``the simplest explanation that sufficiently explains the
data should be preferred over more complex alternatives.'' This is a
nice guiding principle which might be useful in your research,
especially when you are faced with multiple explanations for a
phenomenon. The principle suggests that the simplest explanation is
often the best one, and that more complex explanations should only be
considered when the simpler ones fail to account for the data. But, keep
in mind that biological systems tend to be complex, and oversimplifying
an explanation may ignore important interactions or heterogeneities.

\subsection{Question 4}\label{question-4}

Explain the difference between R and RStudio.

\textbf{{[}5 marks{]}}

\textbf{Answer}

Taken verbatim from Tangled Bank:

\textbf{R is a programming language and software environment for
statistical computing and graphics}. It provides a wide variety of
statistical (linear and non-linear modelling, classical statistical
tests, time-series analysis, classification, clustering, multivariate
analyses, neural networks, and so forth) and graphical techniques, and
is highly extensible.

\textbf{RStudio is an integrated development environment (IDE)} for R.
It provides a graphical user interface (GUI) for working with R, making
it easier to use for those who are less familiar with command-line
interfaces. Some of the features provided by RStudio include:

\begin{itemize}
\tightlist
\item
  a code editor with syntax highlighting and code completion;
\item
  a console for running R code;
\item
  a graphical interface for managing packages and libraries;
\item
  an integrated tools for plotting and visualisation; and
\item
  support for version control with Git and SVN.
\end{itemize}

R is the core software for statistical computing, like a car's engine,
while RStudio provides a more user-friendly interface for working with
R, like the car's body, the seats, steering wheel, and other bells and
whistles.

\subsection{Question 5}\label{question-5}

By way of example, please explain some key aspects of R code
conventions. For each line of code, explain also in English what aspects
of the code are being adhered to.

For example:

\begin{enumerate}
\def\labelenumi{\arabic{enumi}.}
\tightlist
\item
  \texttt{a\ \textless{}-\ b} is not the same as
  \texttt{a\ \textless{}\ -b}. The former is correct because there is a
  space preceding and following the assignment operator
  (\texttt{\textless{}-}, a less-than sign immediately followed by a
  dash to form an arrow); this has a different meaning from the latter,
  which is incorrect because there is no space between the less-than
  sign and the dash, reading as ``a is less than negative b''.
\end{enumerate}

{Hint: In your Word document, use a fixed-width font to indicate the
code as a separate block which is distinct from the rest of the text.}

\textbf{{[}10 marks{]}}

\textbf{Answer}

\begin{enumerate}
\def\labelenumi{\arabic{enumi}.}
\tightlist
\item
  Proper use of indentation:
\end{enumerate}

\begin{Shaded}
\begin{Highlighting}[]
\ControlFlowTok{if}\NormalTok{ (x }\SpecialCharTok{\textgreater{}} \DecValTok{0}\NormalTok{) \{}
  \FunctionTok{print}\NormalTok{(}\StringTok{"Positive number"}\NormalTok{)}
\NormalTok{\}}
\end{Highlighting}
\end{Shaded}

\begin{enumerate}
\def\labelenumi{\arabic{enumi}.}
\setcounter{enumi}{1}
\tightlist
\item
  Use of meaningful variable names:
\end{enumerate}

\begin{Shaded}
\begin{Highlighting}[]
\NormalTok{temperature }\OtherTok{\textless{}{-}} \DecValTok{25}
\end{Highlighting}
\end{Shaded}

\begin{enumerate}
\def\labelenumi{\arabic{enumi}.}
\setcounter{enumi}{2}
\tightlist
\item
  Use of comments to explain code:
\end{enumerate}

\begin{Shaded}
\begin{Highlighting}[]
\CommentTok{\# Calculate the mean temperature}
\NormalTok{mean\_temp }\OtherTok{\textless{}{-}} \FunctionTok{mean}\NormalTok{(temperature)}
\end{Highlighting}
\end{Shaded}

\begin{enumerate}
\def\labelenumi{\arabic{enumi}.}
\setcounter{enumi}{3}
\tightlist
\item
  Consistent use of spacing around operators:
\end{enumerate}

\begin{Shaded}
\begin{Highlighting}[]
\NormalTok{a }\OtherTok{\textless{}{-}}\NormalTok{ b }\SpecialCharTok{+}\NormalTok{ c}
\end{Highlighting}
\end{Shaded}

\begin{enumerate}
\def\labelenumi{\arabic{enumi}.}
\setcounter{enumi}{4}
\tightlist
\item
  Consistent use of compound object names:
\end{enumerate}

A principles of writing clean and readable R code (or \emph{any} code)
is maintaining consistent variable naming conventions throughout a
script or project. Mixing different naming styles -- such as
``snake\_case'' (words separated by underscores) and ``camelCase''
(capitalising the first letter of each subsequent word) -- makes the
code harder to read, maintain, and debug.

Examples:

\begin{Shaded}
\begin{Highlighting}[]
\CommentTok{\# Example of consistent use of either convention:}
\NormalTok{my\_variable }\OtherTok{\textless{}{-}} \DecValTok{10} \CommentTok{\# snake case}
\NormalTok{another\_variable }\OtherTok{\textless{}{-}} \DecValTok{20} \CommentTok{\# camel case}

\CommentTok{\# An example of inconsistent use of conventions:}
\NormalTok{myVariable }\OtherTok{\textless{}{-}} \DecValTok{30} \CommentTok{\# camel case}
\NormalTok{yet\_another\_variable }\OtherTok{\textless{}{-}} \DecValTok{40} \CommentTok{\# snake case}

\CommentTok{\# This is also incorrect:}
\NormalTok{variable\_one }\OtherTok{\textless{}{-}} \DecValTok{13} \CommentTok{\# llowercase "one"}
\NormalTok{variable\_Two }\OtherTok{\textless{}{-}} \DecValTok{13} \SpecialCharTok{*} \DecValTok{2} \CommentTok{\# uppercase "Two"}
\end{Highlighting}
\end{Shaded}

\begin{enumerate}
\def\labelenumi{\arabic{enumi}.}
\setcounter{enumi}{5}
\tightlist
\item
  Avoiding the = as Assignment Operator
\end{enumerate}

\begin{Shaded}
\begin{Highlighting}[]
\CommentTok{\# Correct:}
\NormalTok{a }\OtherTok{\textless{}{-}} \DecValTok{1}

\CommentTok{\# Incorrect:}
\NormalTok{a }\OtherTok{=} \DecValTok{1}
\end{Highlighting}
\end{Shaded}

\begin{enumerate}
\def\labelenumi{\arabic{enumi}.}
\setcounter{enumi}{6}
\tightlist
\item
  Consistent use of spaces around \# symbols in comments:
\end{enumerate}

\begin{Shaded}
\begin{Highlighting}[]
\CommentTok{\# This is correct:}

\CommentTok{\# This is a comment}
\CommentTok{\# This is another comment}
\CommentTok{\# And another}

\CommentTok{\# This is incorrect:}

\CommentTok{\#This is a comment}
\CommentTok{\# A comment?}
\CommentTok{\#  Another comment}
\end{Highlighting}
\end{Shaded}

\begin{enumerate}
\def\labelenumi{\arabic{enumi}.}
\setcounter{enumi}{7}
\tightlist
\item
  Correct use of \texttt{+} or \texttt{-} for unary operators:
\end{enumerate}

\begin{Shaded}
\begin{Highlighting}[]
\CommentTok{\# Correct:}
\NormalTok{a }\OtherTok{\textless{}{-}} \SpecialCharTok{{-}}\NormalTok{b}
\end{Highlighting}
\end{Shaded}

\begin{enumerate}
\def\labelenumi{\arabic{enumi}.}
\setcounter{enumi}{8}
\tightlist
\item
  Use of \texttt{TRUE} and \texttt{FALSE} instead of \texttt{T} and
  \texttt{F}:
\end{enumerate}

\begin{Shaded}
\begin{Highlighting}[]
\CommentTok{\# Correct:}
\NormalTok{is\_positive }\OtherTok{\textless{}{-}} \ConstantTok{TRUE}

\CommentTok{\# Incorrect:}
\NormalTok{is\_positive }\OtherTok{\textless{}{-}}\NormalTok{ T}
\end{Highlighting}
\end{Shaded}

For more, refer to the
\href{https://style.tidyverse.org/syntax.html}{tidyverse style guide}.

\subsection{Question 6}\label{question-6}

\begin{enumerate}
\def\labelenumi{\alph{enumi}.}
\tightlist
\item
  Explain why one typically prefers working with CSV files over Excel
  files in R.
\item
  What are the properties of a CSV file that make it more suitable for
  data analysis in R?
\item
  What are the properties of an Excel file that make it less suitable
  for data analysis in R?
\end{enumerate}

\textbf{{[}15 marks{]}}

\textbf{Answer}

\begin{enumerate}
\def\labelenumi{\alph{enumi})}
\tightlist
\item
\end{enumerate}

CSV (Comma-Separated Values) files are preferred over Excel files due to
their simplicity, compatibility, and efficiency in handling data. CSV
files are stored as plain text, making them easy to read and write
across different software and platforms. They do not contain proprietary
formatting, formulas, or metadata, which minimises the risk of
unintended data transformations.

Excel files (.xls, .xlsx) are proprietary and designed for spreadsheet
applications, incorporating complex formatting, formulas, and visual
formatting that can interfere with data processing in R. Unlike CSV
files, which can be directly read using base R functions like
\texttt{read.csv()}, Excel files require additional packages such as
\textbf{readxl} for data extraction. Excel's tendency to automatically
modify data types -- such as converting text to dates or numbers -- is
annoying and introduces errors, making CSV a more reliable format for
reproducible data analysis.

\begin{enumerate}
\def\labelenumi{\alph{enumi})}
\setcounter{enumi}{1}
\tightlist
\item
\end{enumerate}

\begin{itemize}
\tightlist
\item
  CSV files store data in a simple text-based format that ensures easy
  readability by both humans and computers.
\item
  Each row represents a single record, and fields are separated by
  commas (or another delimiter) to ensure a consistent tabular format.
\item
  CSV files can be opened and edited using a wide range of software,
  including text editors, spreadsheets (e.g., Excel, Google Sheets), and
  statistical tools (e.g., R, Python).
\item
  R provides optimised functions like \texttt{read.csv()} (base R) and
  \texttt{read\_csv()} (tidyverse) for quickly reading CSV files without
  additional dependencies.
\item
  Unlike Excel, CSV files do not contain embedded formulas, formatting,
  figures, or macros and these properties reduce the risk of unintended
  data stuff-ups.
\item
  Being plain text, CSV files are typically smaller in size compared to
  Excel files.
\end{itemize}

\begin{enumerate}
\def\labelenumi{\alph{enumi})}
\setcounter{enumi}{2}
\tightlist
\item
\end{enumerate}

\begin{itemize}
\tightlist
\item
  Excel files are stored in a format (.xls, .xlsx) that is specific to
  Microsoft Excel; special packages (e.g., \textbf{readxl},
  \textbf{openxlsx}) are needed to read them in R.
\item
  Excel often automatically formats data and changes numeric values to
  dates or rounding decimal values. This can lead to errors in data
  analysis.
\item
  Excel files support formulas, pivot tables, conditional formatting,
  and visual elements that may not be relevant for raw data processing
  in R.
\item
  Users can store multiple sheets within a single Excel file and this
  makes it trickier to maintain a standardised structure when importing
  data into R.
\item
  Excel files are not made for handling large datasets. Excel becomes
  very slow and is prone to crashing or memory limitations when dealing
  with `big' data.
\item
  Excel's binary files do not work with version control systems like
  Git.
\item
  Excel files are complex and more prone to accidental modifications or
  corruption.
\end{itemize}

\subsection{Question 7}\label{question-7}

Explain each of the following in the context of their use in R. For
each, provide an example of how you would construct them in R:

\begin{enumerate}
\def\labelenumi{\alph{enumi}.}
\tightlist
\item
  A vector
\item
  A matrix
\item
  A dataframe
\item
  A list
\end{enumerate}

Hint: See my hint under Question 5.

\textbf{{[}20 marks{]}}

\textbf{Answer}

\begin{enumerate}
\def\labelenumi{(\alph{enumi})}
\tightlist
\item
  A vector in R is the simplest and most fundamental data structure. It
  is a one-dimensional collection of elements, all of the same type
  (e.g., numeric, character, or logical). Vectors can be created using
  the \texttt{c()} function. For example:
\end{enumerate}

\begin{Shaded}
\begin{Highlighting}[]
\CommentTok{\# Creating a numeric vector}
\NormalTok{numbers }\OtherTok{\textless{}{-}} \FunctionTok{c}\NormalTok{(}\DecValTok{1}\NormalTok{, }\DecValTok{2}\NormalTok{, }\DecValTok{3}\NormalTok{, }\DecValTok{4}\NormalTok{, }\DecValTok{5}\NormalTok{)}

\CommentTok{\# Creating a character vector}
\NormalTok{names }\OtherTok{\textless{}{-}} \FunctionTok{c}\NormalTok{(}\StringTok{"Acacia"}\NormalTok{, }\StringTok{"Protea"}\NormalTok{, }\StringTok{"Leucadendron"}\NormalTok{)}

\CommentTok{\# Creating a logical vector}
\NormalTok{logical\_values }\OtherTok{\textless{}{-}} \FunctionTok{c}\NormalTok{(}\ConstantTok{TRUE}\NormalTok{, }\ConstantTok{FALSE}\NormalTok{, }\ConstantTok{TRUE}\NormalTok{)}
\end{Highlighting}
\end{Shaded}

\begin{enumerate}
\def\labelenumi{(\alph{enumi})}
\setcounter{enumi}{1}
\tightlist
\item
  A matrix is a two-dimensional data structure where all elements must
  be of the same type. It is essentially an extension of a vector with a
  specified number of rows and columns.
\end{enumerate}

\begin{Shaded}
\begin{Highlighting}[]
\CommentTok{\# Creating a matrix with 3 rows and 2 columns}
\NormalTok{my\_matrix }\OtherTok{\textless{}{-}} \FunctionTok{matrix}\NormalTok{(}\FunctionTok{c}\NormalTok{(}\DecValTok{1}\NormalTok{, }\DecValTok{2}\NormalTok{, }\DecValTok{3}\NormalTok{, }\DecValTok{4}\NormalTok{, }\DecValTok{5}\NormalTok{, }\DecValTok{6}\NormalTok{), }\AttributeTok{nrow =} \DecValTok{3}\NormalTok{, }\AttributeTok{ncol =} \DecValTok{2}\NormalTok{)}
\end{Highlighting}
\end{Shaded}

\begin{enumerate}
\def\labelenumi{(\alph{enumi})}
\setcounter{enumi}{2}
\tightlist
\item
  A dataframe is a two-dimensional data structure that can contain
  different data types in different columns (variables). It is the most
  commonly used data structure for data analysis in R and resembles a
  table with rows and columns.
\end{enumerate}

\begin{Shaded}
\begin{Highlighting}[]
\CommentTok{\# Creating a dataframe}
\NormalTok{my\_dataframe }\OtherTok{\textless{}{-}} \FunctionTok{data.frame}\NormalTok{(}
  \AttributeTok{Name =} \FunctionTok{c}\NormalTok{(}\StringTok{"Acacia"}\NormalTok{, }\StringTok{"Protea"}\NormalTok{, }\StringTok{"Leucadendron"}\NormalTok{),}
  \AttributeTok{Age =} \FunctionTok{c}\NormalTok{(}\DecValTok{25}\NormalTok{, }\DecValTok{30}\NormalTok{, }\DecValTok{22}\NormalTok{),}
  \AttributeTok{Height =} \FunctionTok{c}\NormalTok{(}\FloatTok{85.5}\NormalTok{, }\FloatTok{90.3}\NormalTok{, }\FloatTok{78.0}\NormalTok{)}
\NormalTok{)}
\end{Highlighting}
\end{Shaded}

\begin{enumerate}
\def\labelenumi{(\alph{enumi})}
\setcounter{enumi}{3}
\tightlist
\item
  A list is a flexible data structure that can store elements of
  different types, including vectors, matrices, dataframes, and even
  other lists. Unlike vectors and matrices, which require uniform data
  types, lists can contain heterogeneous elements.
\end{enumerate}

\begin{Shaded}
\begin{Highlighting}[]
\CommentTok{\# Creating a list with different data types}
\CommentTok{\# Uses the data created abobe, for example}
\NormalTok{my\_list }\OtherTok{\textless{}{-}} \FunctionTok{list}\NormalTok{(}
  \AttributeTok{plants =}\NormalTok{ my\_dataframe,}
  \AttributeTok{some\_numbers =}\NormalTok{ mu\_matrix,}
  \AttributeTok{other\_numbers =}\NormalTok{ numbers}
\NormalTok{  )}
\end{Highlighting}
\end{Shaded}

\subsection{Question 8}\label{question-8}

\begin{enumerate}
\def\labelenumi{\alph{enumi}.}
\item
  Write a 150 to 200 word abstract about your Honours research project.
  In your abstract, draw attention to the types of data you will be
  expected to generate, and mention how these will be used to address
  your research question.
\item
  Explain which of the R data classes will be most useful in your
  research and why.
\item
  With reference to the abstract you wrote in Question 8.a, explain how
  you would visualise (or display your finding in tabular format) your
  research findings. Provide an example of how you would do this in R.
  Which of your research questions would be best answered using a
  visualisations or tables? What do you expect your visualisations or
  tables to show?
\item
  Provide an example of how you would create a plot or table in R.
  Generate mock code (it does not need to run) that you would use to
  create the plot or table.
\end{enumerate}

Note 1: In the unlikely event that your research will not require
visualisations or tables, please explain why this is the case and how
you would communicate your findings.

Note 2: If you haven't defined your research project yet, describe a
hypothetical project in your field of interest.

\textbf{{[}30 marks{]}}

\textbf{Answer}

This will have to be assessed based on the information quality produced
in each abstract. Assign marks as follows:

\begin{enumerate}
\def\labelenumi{\alph{enumi}.}
\tightlist
\item
  Abstract: 30\%
\item
  Data classes: 10\%
\item
  Visualisation: 30\%
\item
  Mock code: 30\%
\end{enumerate}

\textbf{TOTAL MARKS: 110}

\section{Practical Test}\label{practical-test}

{\textbf{This is the open book assessment.}}

Below is a set of scripting problems to solve. You have 21 hours from
the end of the Theory Test to complete this section Please write your
code in an R script file and submit it to the iKamva platform by no
later than 8:30 on Tuesday, 18 March 2025.

Please follow a clear structure (appropriate, clearly numbered headings
and subheadings) in your code, including comments and explanations.

{Ensure that all code runs without errors before submitting it --
serious penalties will apply to non-functional scripts.}

Naming convention: \texttt{Intro\_R\_Test\_Practical\_YourSurname.R}

\subsection{Question 1}\label{question-1-1}

Download the
\href{https://raw.githubusercontent.com/ajsmit/R_courses/main/static/data/fertiliser_crop_data.csv}{\texttt{fertiliser\_crop\_data.csv}}
data.

The data represent an experiment designed to test whether or not
fertiliser type and the density of planting have an effect on the yield
of wheat. The dataset contains the following variables:

\begin{itemize}
\tightlist
\item
  Final yield (kg per acre) -- make sure to convert this to the most
  suitable SI unit before continuing with your analysis
\item
  Type of fertiliser (fertiliser type A, B, or C)
\item
  Planting density (1 = low density, 2 = high density)
\item
  Block in the field (north, east, south, west)
\end{itemize}

Undertake a full visual assessment of the dataset and establish which of
the influential variables are most likely to have an effect on crop
yield. Provide a detailed explanation of your findings.

\textbf{{[}25 marks{]}}

\textbf{Answer}

\subsection{Question 2}\label{question-2-1}

The Bullfrog Occupancy and Common Reed Invasion data are here:
\texttt{AICcmodavg::bullfrog} (i.e.~the \texttt{bullfrogs} dataset
resides within the \textbf{AICcmodavg} package, which you might have to
install).

Create a tidy dataframe from the bullfrog data.

\textbf{{[}10 marks{]}}

\textbf{Answer}

\subsection{Question 3}\label{question-3-1}

The Growth Curves for Sitka Spruce Trees in 1988 and 1989 data are here:
\texttt{MASS::Sitka} and \texttt{MASS::Sitka89}.

Combine the two datasets and provide an analysis of the growth curves
for Sitka spruce trees in 1988 and 1989. Give graphical support for the
idea that i) ozone affects the growth of Sitka spruce trees, and ii) the
growth of Sitka spruce trees is affected by the year of measurement. In
addition to showing the overall response in each year x treatment, also
ensure that the among tree variability is visible.

Explain your findings.

\textbf{{[}20 marks{]}}

\textbf{Answer}

\subsection{Question 4}\label{question-4-1}

The Frog Dehydration Experiment on Three Substrate Types data can be
accessed here: \texttt{AICcmodavg::dry.frog}.

\begin{enumerate}
\def\labelenumi{\alph{enumi}.}
\tightlist
\item
  Based on the dataset, what do you think was the purpose of this study?
  Provide a 200 word synopsis as your answer.
\item
  Create new columns in the dataframe showing:

  \begin{itemize}
  \tightlist
  \item
    the final mass;
  \item
    the percent mass lost; and
  \item
    the percent mass lost as a function of the initial mass of each
    frog.
  \end{itemize}
\item
  Provide the R code that would have resulted in the data in the
  variables \texttt{cent\_initial\_mass} and \texttt{cent\_Air}.
\item
  An analysis of the factors responsible for dehydration rates in frogs.
  In your analysis, consider the effects substrate type, initial mass,
  air temperature, and wind.
\item
  Provide a brief discussion of your findings.
\end{enumerate}

\textbf{{[}25 marks{]}}

\textbf{Answer}

\subsection{Question 5}\label{question-5-1}

Consider this script:

\begin{Shaded}
\begin{Highlighting}[]
\FunctionTok{ggplot}\NormalTok{(points, }\FunctionTok{aes}\NormalTok{(}\AttributeTok{x =}\NormalTok{ group, }\AttributeTok{y =}\NormalTok{ count)) }\SpecialCharTok{+}
  \FunctionTok{geom\_boxplot}\NormalTok{(}\FunctionTok{aes}\NormalTok{(}\AttributeTok{colour =}\NormalTok{ group), }\AttributeTok{size =} \DecValTok{1}\NormalTok{, }\AttributeTok{outlier.colour =} \ConstantTok{NA}\NormalTok{) }\SpecialCharTok{+}
  \FunctionTok{geom\_point}\NormalTok{(}\AttributeTok{position =} \FunctionTok{position\_jitter}\NormalTok{(}\AttributeTok{width =} \FloatTok{0.2}\NormalTok{), }\AttributeTok{alpha =} \FloatTok{0.3}\NormalTok{) }\SpecialCharTok{+}
  \FunctionTok{facet\_grid}\NormalTok{(group }\SpecialCharTok{\textasciitilde{}}\NormalTok{ ., }\AttributeTok{scales =} \StringTok{"free"}\NormalTok{) }\SpecialCharTok{+}
  \FunctionTok{labs}\NormalTok{(}\AttributeTok{x =} \StringTok{""}\NormalTok{, }\AttributeTok{y =} \StringTok{"Number of data points"}\NormalTok{) }\SpecialCharTok{+}
  \FunctionTok{theme}\NormalTok{(}\AttributeTok{legend.position =} \StringTok{"none"}\NormalTok{,}
    \AttributeTok{strip.background =} \FunctionTok{element\_blank}\NormalTok{(),}
    \AttributeTok{strip.text =} \FunctionTok{element\_blank}\NormalTok{())}
\end{Highlighting}
\end{Shaded}

\begin{enumerate}
\def\labelenumi{\alph{enumi}.}
\tightlist
\item
  Generate fictitious (random, normal) data that can be plotted using
  the code, above. Make sure to assemble these data into a dataframe
  suitable for plotting, complete with correct column titles.
\item
  Apply the script \emph{exactly as stated} to the data to demonstate
  your understanding of the code and convince the examiner of your
  understanding of the correct data structure.
\end{enumerate}

\textbf{{[}10 marks{]}}

\textbf{Answer}

\subsection{Question 6}\label{question-6-1}

For this assessment, you will analyse the built-in R dataset
\texttt{datasets::UKDriverDeaths}, which contains monthly totals of car
drivers killed or seriously injured in road accidents in Great Britain
from January 1969 to December 1984. This time series data allows for
examination of long-term trends, seasonal patterns, and potential
correlations with societal factors.

\begin{enumerate}
\def\labelenumi{\alph{enumi}.}
\tightlist
\item
  \textbf{Data Exploration and Preparation}

  \begin{enumerate}
  \def\labelenumii{\roman{enumii}.}
  \tightlist
  \item
    Load the \texttt{UKDriverDeaths} dataset and examine its structure.
    Convert the time series data into a standard data frame format with
    separate columns for:

    \begin{itemize}
    \tightlist
    \item
      Year
    \item
      Month (both as a number and as a factor with proper names)
    \item
      Number of deaths/injuries
    \end{itemize}
  \item
    Create a new variable that classifies each month into seasons
    (Winter: Dec-Feb, Spring: Mar-May, Summer: Jun-Aug, Autumn:
    Sep-Nov).
  \item
    Create another variable identifying whether each observation falls
    during a major energy crisis period (e.g., the oil crises of
    1973-1974 and 1979-1980).
  \item
    Identify and handle any potential inconsistencies or issues in the
    dataset that might affect subsequent analyses.
  \end{enumerate}
\end{enumerate}

\textbf{{[}20 marks{]}}

\textbf{Answer}

\begin{enumerate}
\def\labelenumi{\alph{enumi}.}
\setcounter{enumi}{1}
\tightlist
\item
  \textbf{Temporal Trend Analysis}

  \begin{enumerate}
  \def\labelenumii{\roman{enumii}.}
  \tightlist
  \item
    Create a comprehensive visualisation showing the full time series
    with:

    \begin{itemize}
    \tightlist
    \item
      Clear temporal trends
    \item
      A smoothed trend line
    \item
      Vertical lines or shading indicating major UK policy changes
      related to road safety (e.g., 1983 seat belt law)
    \item
      Annotations for key events
    \end{itemize}
  \item
    Develop a visualisation examining monthly fatality averages across
    the entire period, ordered appropriately to show seasonal patterns.
  \item
    Create a visualisation that compares annual patterns between the
    first half of the dataset (1969-1976) and the second half
    (1977-1984).
  \item
    Using \emph{tidy} data manipulation techniques, calculate and
    visualise the year-over-year percent change in fatalities for each
    month throughout the dataset.
  \end{enumerate}
\end{enumerate}

\textbf{{[}20 marks{]}}

\textbf{Answer}

\begin{enumerate}
\def\labelenumi{\alph{enumi}.}
\setcounter{enumi}{2}
\tightlist
\item
  \textbf{Pattern Analysis and Decomposition}

  \begin{enumerate}
  \def\labelenumii{\roman{enumii}.}
  \tightlist
  \item
    Calculate and visualise the average number of fatalities by season
    across all years.
  \item
    Create a heatmap showing fatalities by month and year, with
    appropriate color scaling to highlight temporal clusters.
  \item
    Implement a decomposition of the time series to separate: - The
    overall trend - Seasonal patterns - Remaining variation
  \item
    Visualise each component and discuss what factors might contribute
    to the patterns observed.
  \end{enumerate}
\end{enumerate}

Note: Some of this will be new to you. But don't worry, use any means
available to you to solve the problem.

\textbf{{[}25 marks{]}}

\textbf{Answer}

\begin{enumerate}
\def\labelenumi{\alph{enumi}.}
\setcounter{enumi}{3}
\tightlist
\item
  \textbf{Data manipulation}
\end{enumerate}

Starting with the data as presented in the \texttt{UKDriverDeaths}
dataset, create a new dataframe identical to the \texttt{Seatbelts}
dataset.

\textbf{{[}5 marks{]}}

\textbf{Answer}

\textbf{TOTAL MARKS: 160}

\textbf{-- THE END --}




\end{document}
