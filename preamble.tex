% preamble.tex

% --- Document Structure and Layout ---

\usepackage[a4paper, total={6in, 8in}]{geometry}

% --- Paragraph Settings ---

\setlength{\parindent}{0pt}
\setlength{\parskip}{6pt}

% --- Fonts and Encoding ---

\usepackage{fontspec}         % Allows font specification
% \usepackage{amsmath}          % For math symbols

%% Main Font
\setmainfont[
  Renderer=Basic, % Recommended for plain text
  UprightFont = MinionPro-Regular,
  ItalicFont = MinionPro-It,
  ItalicFeatures = { SmallCapsFont = MinionPro-It },
  SlantedFont = MinionPro-Regular,
  SlantedFeatures= { FakeSlant=0.2 },
  BoldFont = MinionPro-Bold,
  BoldFeatures = { SmallCapsFont = MinionPro-Bold },
  BoldItalicFont = MinionPro-BoldIt,
  BoldItalicFeatures = { SmallCapsFont = MinionPro-BoldIt },
  BoldSlantedFont= MinionPro-Bold,
  BoldSlantedFeatures= { FakeSlant=0.2, SmallCapsFont = MinionPro-Bold },
  SmallCapsFont = MinionPro-Regular,
  SmallCapsFeatures={ RawFeature=+smcp },
  Ligatures=TeX,
  Numbers={OldStyle, Proportional}
]{MinionPro-Regular}

%% Math Font
\setmathfont{MinionPro-Regular.otf}

%% Monospace Font
\setmonofont[
  Scale=0.89
]{FiraCode Nerd Font}
% Define a command to apply the small font size to all monospaced content
\renewcommand{\ttfamily}{\small\fontspec{FiraCode Nerd Font}} % Use \small consistently

% --- Packages for Tables ---

\usepackage{array}            % For table column width specification
\usepackage{booktabs}         % For table rules
\usepackage{ragged2e}         % For text alignment (used with \newcolumntype)

% --- Headers and Footers ---

\usepackage{fancyhdr}
\pagestyle{fancy}
\renewcommand{\sectionmark}[1]{\markright{#1}{}}
\fancyhf{}
\fancyhead[LE,RO]{\thepage}
\fancyhead[LO]{\textsc{\MakeLowercase{\leftmark}}}
\fancyhead[RE]{\textsc{\MakeLowercase{\rightmark}}}

% --- Other Packages ---

\usepackage[version=4]{mhchem}% Formula subscripts using \ce{}

% --- Color Definitions ---

\usepackage[x11names]{xcolor} % Required for specifying custom colors, load before tcolorbox
\definecolor{headingblue}{RGB}{23,48,191}
\definecolor{boxtitle}{HTML}{F0F4F8}
\definecolor{boxbody}{HTML}{FBFDFF}
\definecolor{mainboxframe}{HTML}{F0F4F8}
\definecolor{subboxframe}{HTML}{F0F4F8}
\definecolor{crimson}{HTML}{880000}

% --- Key Terms

\newcommand{\keyterm}[1]{\textsc{#1}}

%% Create a command for color emphasis
\newcommand{\highlight}[1]{\textcolor{crimson}{#1}}

% --- Boxes ---

% Define the mdframed environment
\usepackage{float}
\usepackage{mdframed}

% 1) Define a new float environment called "boxfloat"
\newfloat{boxfloat}{htbp}{lob}
\floatname{boxfloat}{Box}

% 3) Define the environment that wraps mdframed in a float
\newenvironment{boxedfloat}[2][]{%
  % Advance the box counter to produce "Chapter.BoxNo"
  \refstepcounter{boxcounter}%
  % Begin the float environment
  \begin{boxfloat}[htbp]
  % Begin the mdframed styling
  \begin{mdframed}[
    backgroundcolor=gray!5,
    innertopmargin=6pt,
    innerbottommargin=6pt,
    innerrightmargin=6pt,
    innerleftmargin=6pt,
    linewidth=0.25pt,
    linecolor=black,
    roundcorner=8pt,    % or 0pt if you prefer sharp corners
    skipabove=12pt,     % vertical space above the box
    skipbelow=12pt,     % vertical space below the box
    innermargin=0pt,
    outermargin=0pt
  ]%
    % Typeset the box heading: "Box 1.1. My Title"
    \setlength{\parindent}{0em}%
    \setlength{\parskip}{3pt}%
    \RaggedRight
    % Both "Box" and the user-supplied title are in small caps
    \small% switch the box contents to smaller text
    {\scshape Box \theboxcounter. #2}\par
    \vspace{6pt} % a little space after the heading
}{%
    \end{mdframed}
    \end{boxfloat}
}

% --- sansblock Environment ---

\usepackage{sourcesanspro}    % Load Source Sans Pro
\setsansfont{Source Sans Pro} % Set it as the sans-serif font
\newenvironment{sansblock}[1]
    {\small\sffamily\raggedright{\scshape #1}\ } % Ensure small caps for the title
  {} % End environment: no special commands needed

% --- Custom Column Type (using ragged2e) ---

\newcolumntype{R}[1]{>{\RaggedRight}p{#1}}

% ---  Margin Notes ---

\usepackage{marginnote}
\renewcommand*{\marginfont}{\footnotesize\itshape} % Style for margin notes

%% Set margin note outer margin to 0.7in
\setlength{\marginparwidth}{1.25in}

% --- Epigraph ---

\usepackage{epigraph}
\setlength\epigraphwidth{.9\textwidth}
\newenvironment{quotepara}
  {\itshape\raggedright\small\setlength{\parskip}{0.5em}} % Add small space between paragraphs
  {}
\renewcommand{\textflush}{quotepara}

%% Define a new epigraph environment without the horizontal rule and source/author
\newenvironment{simpleepigraph}
  {\begin{list}{}%
      {\setlength{\leftmargin}{2em}% Left margin
       \setlength{\rightmargin}{2em}% Right margin
       \setlength{\topsep}{1em}% Space above the epigraph
       \setlength{\itemsep}{0pt}% Space between items (irrelevant here)
       \setlength{\parsep}{0pt}}% Space between paragraphs
   \item\relax\raggedright\small} % Apply ragged-right and italic style for the epigraph text
  {\end{list}}

% --- Small Caps ---

\newcommand{\flatcaps}[1]{\textsc{\MakeLowercase{#1}}}

% --- Lists ---

%% General settings for all lists
\usepackage{enumitem}

% Global settings following Bringhurst's principles
% A global default to keep lists tight, but still allow subtle spacing:
\setlist{
  nosep,         % No extra space between items
  topsep=0.6em,  % A bit of space before/after the list
  parsep=0pt,
  partopsep=0pt
}

% First-level itemize (unordered) lists:
\setlist[itemize,1]{
  label=\textbullet,
  labelsep=0.4em,        % Space from bullet to text
  labelwidth=1em,        % Horizontal space set aside for bullet
  leftmargin=\dimexpr 1em + 0.4em\relax,
  itemindent=0pt,
  listparindent=0pt,
  align=left
}

% Second-level itemize, with a subtler symbol:
\setlist[itemize,2]{
  label=--,
  labelsep=0.4em,
  labelwidth=1em,
  leftmargin=\dimexpr 1em + 0.4em\relax,
  itemindent=0pt,
  listparindent=0pt,
  align=left
}

% First-level enumerate (ordered) lists:
\setlist[enumerate,1]{
  label=\arabic*.,
  labelsep=0.4em,
  labelwidth=1em,
  leftmargin=\dimexpr 1em + 0.4em\relax,
  itemindent=0pt,
  listparindent=0pt,
  align=left
}

% Second-level enumerate (letters, or you could do roman numerals):
\setlist[enumerate,2]{
  label=\alph*.,
  labelsep=0.4em,
  labelwidth=1em,
  leftmargin=\dimexpr 1em + 0.4em\relax,
  itemindent=0pt,
  listparindent=0pt,
  align=left
}

% --- Custom Chapter/Section Styles ---

\usepackage[compact]{titlesec} % Allows creating custom chapter styles
\titleformat{\chapter}[display]
  {\fontsize{60}{62}\bfseries}
  {\thechapter}
  {0pt}
  {\huge\noindent}
\titlespacing*{\chapter}{0pt}{0pt}{40pt}

\titleformat{\section}
  {\normalsize\normalfont}
  {\thesection}
  {0.6em}
  {\flatcaps}
\titlespacing*{\section}{0pt}{1\baselineskip}{1\baselineskip}

\titleformat{\subsection}[block]
  {\normalsize\normalfont} % defines the font size and style for the entire subsection heading, including both the number and the title
  {\thesubsection} % defines the format of the subsection number
  {1em} % the horizontal space between the subsection number and the title
  {\itshape} % defines the format of the subsection title itself
\titlespacing*{\subsection}{0pt}{1\baselineskip}{1\baselineskip}

\titleformat{\subsubsection}[runin]
  {\normalsize\normalfont} % defines the font size and style for the entire subsection heading, including both the number and the title
  {\thesubsubsection} % defines the format of the subsection number
  {1em} % the horizontal space between the subsection number and the title
  {\itshape}[.] % defines the format of the subsection title itself
\titlespacing*{\subsubsection}{0pt}{1\baselineskip}{1\baselineskip}

\titleformat{\paragraph}[runin]
  {\flatcaps}
  {\theparagraph}
  {0pt}
  {}

% --- Footnotes ---

\usepackage[norule,ragged,hang]{footmisc}  % Load footmisc with ragged option
\renewcommand{\footnotelayout}{\RaggedRight\footnotesize} % Typeset footnotes in \RaggedRight
\setlength{\footnotemargin}{1.5em}    % Adjust space between number and text
\makeatletter
\renewcommand{\@makefntext}[1]{%
    \parindent 1em%                    Set parindent for footnote text
    \noindent
    \hb@xt@ 1.8em{%                   Set hanging indent for footnote text
        \hss\@thefnmark.%
    }
    \RaggedRight #1%                 Typeset footnote text ragged right
}
\makeatother

% --- Captions ---

\usepackage{caption}
\captionsetup{
  font={small},
  labelfont={bf},
  textfont={},
  width=0.9\textwidth,
  justification=justified,
  labelformat=default,
  labelsep=period,
  format=plain
}
\renewcommand{\captionlabelfont}{\bfseries\scshape}

% --- Hyperlinks ---

\usepackage{hyperref}           % Load after most other packages, but before cleveref
\hypersetup{
    colorlinks=true,
    linkcolor=blue,
    filecolor=magenta,
    urlcolor=cyan,
    pdftitle={Overleaf Example},
    pdfpagemode=FullScreen,
    }

\urlstyle{same}

% --- Miscellaneous ---

%% Define a new command to print the current page number to the console
\ifluatex
  \usepackage{luacode}
  \usepackage{shellesc}
  \newcommand{\printpagenumber}{%
    \directlua{
      local pagenumber = tex.count.page
      print(string.format("Currently processing page: %d", pagenumber))
    }
  }
\fi

\usepackage{etoolbox}          % General package for patching commands
\usepackage{iftex}             % Detects the engine used
\usepackage{ellipsis}          % Fixes spacing around ellipses
\AddToHook{env/Highlighting/begin}{\small} % Set the code chunk font size globally

%% Use lining fonts
\newcommand\lining{\addfontfeatures{Numbers={Monospaced, Lining}}}
\AtBeginEnvironment{tabular}{\lining} % In tables
\renewcommand{\theequation}{ {\lining\arabic{equation}}} % For equation numbers

% --- Index (if needed) ---

\usepackage{makeidx}
\makeindex

% --- Other Typography Settings ---

\frenchspacing                % Single space after periods
\tolerance=400                % Default is 200; higher values allow more relaxed line-breaking.
\emergencystretch=3em         % Adds additional space to help line-breaking.
\hyphenpenalty=20             % Default is 50; lower values encourage hyphenation.

% --- Microtype Settings (adjust only if needed) ---

\usepackage{microtype}        % Improves typography
\microtypesetup{
   tracking = true,
   protrusion=true,
   expansion=true,
   factor = 1100,
   stretch = 15,
   shrink = 15
}

