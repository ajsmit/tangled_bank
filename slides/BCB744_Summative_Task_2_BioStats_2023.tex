\documentclass[10pt,a4,]{article}
\usepackage[dvipsnames]{xcolor}
% \RequirePackage[l2tabu, orthodox]{nag}
\usepackage[a4paper,text={16.5cm,25.2cm},centering,margin=2.4cm]{geometry}
% \usepackage[left=1.0in,top=1.0in,right=1.0in,bottom=1.0in]{geometry}
\newcommand*{\authorfont}{\fontfamily{phv}\selectfont}
\usepackage{hyperref,amsmath,amssymb,bm,url,enumitem,dcolumn,upquote,framed,alltt,textgreek,xfrac,fixltx2e}
\usepackage[australian]{babel}
\usepackage[compact,small]{titlesec}
\setlength{\parskip}{1.2ex}
\setlength{\parindent}{0em}

\def\tightlist{}

\usepackage{ifxetex}
\ifxetex
 \usepackage{fontspec}
 \defaultfontfeatures{Ligatures=TeX} % To support LaTeX quoting style
 % \defaultfontfeatures{Ligatures=TeX,Numbers={OldStyle}}
 \setmainfont{Minion Pro}
 \setsansfont[Scale=MatchLowercase]{Myriad Pro}
 \setmonofont[Scale=0.78,Color=RubineRed]{Bitstream Vera Sans Mono}
\else
  \usepackage[T1]{fontenc}
  \usepackage[utf8]{inputenc}
  \usepackage{lmodern}
  % \usepackage[full]{textcomp} % directly use the degree (and some other) symbol
\fi

% place after fonts; even better typesetting for improved readability:
\graphicspath{ {figure/} }
\usepackage{tabularx} % for 'tabularx' environment and 'X' column type
\usepackage{ragged2e}  % for '\RaggedRight' macro (allows hyphenation)
\usepackage{siunitx}
    \sisetup{%
        detect-mode,
        group-digits            = false,
        input-symbols           = ( ) [ ] - + < > *,
        table-align-text-post   = false,
        round-mode              = places,
        round-precision         = 3
        }
% \usepackage[font={small, sf}, labelfont=bf]{caption} % tweaking the captions
\usepackage[font={small}, labelfont=bf]{caption} % tweaking the captions
\usepackage[color=yellow, textsize=tiny]{todonotes}
\usepackage{csquotes}
\frenchspacing%

\usepackage{abstract}
\renewcommand{\abstractname}{} % clear the title
\renewcommand{\absnamepos}{empty} % originally center

\renewenvironment{abstract}
{{%
\setlength{\leftmargin}{0mm}
\setlength{\rightmargin}{\leftmargin}%
}%
\relax}
{\endlist}

\makeatletter
\def\@maketitle{%
\newpage
%  \null
%  \vskip 2em%
%  \begin{center}%
\let \footnote \thanks
 {\fontsize{18}{20}\selectfont\raggedright  \setlength{\parindent}{0pt} \@title \par}%
}
%\fi
\makeatother


\setcounter{secnumdepth}{0}





\usepackage{longtable,booktabs}
\setlength\heavyrulewidth{0.1em}
\setlength\lightrulewidth{0.0625em}


\title{BCB744 (BioStats): Summative Task 2  }

\author{\Large AJ
Smit\vspace{0.05in} \newline\normalsize\emph{University of the Western
Cape}  }

\date{}

% PENALTIES
\widowpenalty=1000
\clubpenalty=1000
\doublehyphendemerits=9999 % Almost no consecutive line hyphens
\brokenpenalty=10000 % No broken words across columns/pages
\interfootnotelinepenalty=9999 % Almost never break footnotes

% SECTION, SUBSECETC.TITLES
\usepackage[compact]{titlesec}
\titleformat{\chapter}
  {\normalfont\LARGE\sffamily\bfseries}
  {\thechapter}
  {1em}
  {}
\titleformat{\section}
  {\normalfont\LARGE\sffamily\bfseries}
  {\thesection}
  {1em}
  {}
\titleformat{\subsection}
  {\normalfont\Large\sffamily\bfseries}
  {\thesubsection}
  {1em}
  {}
\titleformat{\subsubsection}
  {\normalfont\large\sffamily\bfseries\slshape}
  {\thesubsubsection}
  {1em}
  {}
% \titlespacing*{<command>}{<left>}{<before-sep>}{<after-sep>}
\titlespacing*{\chapter}
  {0pt}
  {1.2ex plus 1ex minus .2ex}
  {0.5ex plus .1ex minus .1ex}
\titlespacing*{\section}
  {0pt}
  {1.2ex plus 1ex minus .2ex}
  {0.5ex plus .1ex minus .1ex}
\titlespacing*{\subsection}
  {0pt}
  {1.2ex plus 1ex minus .2ex}
  {0.5ex plus .1ex minus .1ex}
\titlespacing*{\subsubsection}
  {0pt}
  {1.2ex plus 1ex minus .2ex}
  {0.5ex plus .1ex minus .1ex}




\newtheorem{hypothesis}{Hypothesis}
\usepackage{setspace}

\makeatletter

\@ifpackageloaded{hyperref}{}{%
\ifxetex
  \usepackage[setpagesize=false, % page size defined by xetex
              unicode=false, % unicode breaks when used with xetex
              xetex]{hyperref}
\else
  \usepackage[colorlinks=true, citecolor=blue, linkcolor=cyan, pdfborder={0 0 0 }, unicode=true]{hyperref} % place after other packages, but before cleveref
\fi
}

\@ifpackageloaded{color}{
    \PassOptionsToPackage{usenames,dvipsnames}{color}
}{%
    \usepackage[usenames,dvipsnames]{color}
}

\usepackage{cleveref} % cleverly cross referencing figures and tables; last package to include
\setcounter{secnumdepth}{2}
\setcounter{tocdepth}{2}

% To use for resetting the numbering of Appendix Tables and Figures:
%\setcounter{table}{0}
%\renewcommand{\thetable}{A\arabic{table}}
%\setcounter{figure}{0}
%\renewcommand{\thefigure}{A\arabic{figure}}

\makeatother
\hypersetup{breaklinks=true,
            bookmarks=true,
            pdfauthor={AJ Smit (University of the Western Cape)},
             pdfkeywords = {},
            pdftitle={BCB744 (BioStats): Summative Task 2},
            colorlinks=true,
            citecolor=blue,
            urlcolor=blue,
            linkcolor=magenta,
            pdfborder={0 0 0}}
\urlstyle{same}  % don't use monospace font for urls


\begin{document}

% \maketitle

{% \usefont{T1}{pnc}{m}{n}
\setlength{\parindent}{0pt}
\thispagestyle{plain}
{\fontsize{18}{20}\selectfont\raggedright
\maketitle  % title \par
}
{
   \vskip 13.5pt\relax \normalsize\fontsize{11}{12}
\textbf{\authorfont AJ Smit} \hskip 15pt \emph{\small University of the
Western Cape}   
}
}



\vskip 6.5pt

\noindent 

\section*{Instructions}

\begin{enumerate}
\item Produce full R scripts, including the meta-information at the top (name, date, purpose, etc.). Include comments explaining the purpose of the various tests/sections as necessary.
\item At the onset, \emph{assume} that all assumptions are met (they might not be!). For each question, i) state the name of the parametric statistical test selected and write a sentence or two to justify your choice, and ii) write down the appropriate $H_{0}$.
\item Test the assumptions. For each statistical test (as per 1.), state the appropriate assumptions to be tested, mention how you will test the assumptions (incl. writing out the $H_{0}$ if necessary, or explaining the principle behind a graphical method), and explain the findings. Given the outcome of the test of assumptions, is your choice of test selected in (1.) still correct? If some/all the assumption tests are violated, how will you proceed? 
\item Proceed with the analysis, and explain the findings in the light of the hypothesis tests stated at the beginning. ANOVAs, correlations, and regressions will require graphical support; this is not necessary for \emph{t}-tests.
\item Pay attention to formatting. 10\% will be allocated to the appearance of the script, including considerations of aspects of the tidiness of the file, the use of appropriate headings, and adherence to code conventions (e.g. spacing etc.).
\end{enumerate}

\section{The wheat yield data}

Please see the file `\textbf{fertiliser\_crop\_data.csv}' for this
dataset. The data represent an experiment designed to test whether or
not fertiliser type and the density of planting have an effect on the
yield of wheat. The dataset contains the following variables:

\begin{itemize}
\item Final yield (kg per acre)
\item Type of fertiliser (fertiliser type A, B, or C)
\item Planting density (1 = low density, 2 = high density)
\item Block in the field (north, east, south, west)
\end{itemize}

\begin{description}
\item[Question 1] Do fertiliser type and planting density affect the yield of wheat? If so, which is the best density to plant wheat at, and which fertiliser produces the best yield?
\item[Question 2] Does it matter if the wheat is planted in portions of the experimental fields that face north, east, south, or west?
\end{description}

\section{The shells data}

See the `\textbf{shells.csv}' file. This dataset contains measurements
of shell widths and lengths of the left and right valves of two species
of mussels, \emph{Aulacomya} sp. and \emph{Choromytilus} sp. Length and
width measurements are presented in mm.

\begin{description}
\item[Question 3] Which species of mussel is the i) widest and ii) longest?
\item[Question 4] Within each species of mussel, are the four different measurements correlated with each other?
\item[Question 5] Considering \emph{Aulacomya} sp. only, use a linear regression to predict the length of the left valve when the width of the left valve is 15 and 17 mm. 
\end{description}

\section{The health data}

These data are in `\textbf{health.csv}'. Inside the file are several
columns, but the ones that are relevant to this question are:

\begin{itemize}
\item 'Sex', which is the gender of the individuals assessed
\item 'Substance', indicating the kind of drug abused by the individuals in question
\item 'Mental\_score', which is the outcome of a test designed to test the cognitive ability of individuals
\end{itemize}

\begin{description}
\item[Question 6] Do males and females suffer the same cognitive impairments if they abuse cocaine, alcohol, or heroin?
\item[Question 7] Which drug is worst in terms of affecting the user's mental health?
\end{description}

\section{The air quality data}

Package \textbf{datasets}, dataset \texttt{airquality}. These are daily
air quality measurements in New York, May to September 1973. See the
help file for details.

\begin{description}
\item[Question 8] Which two of the four response variables are best correlated with each other?
\item[Question 9] Provide a detailed correlation analysis for those two variables.
\end{description}

\section{The crickets data}

The file `\textbf{crickets.csv}' contains data for some crickets whose
chirp rate was measured at several temperatures. The temperature was
measured in °F, but please make sure you do all the calculations using
°C instead.

\begin{description}
\item[Question 10] Does the chirp rate of the crickets depend on the temperature?
\item[Question 11] Provide an equation that quantifies this relationship.
\end{description}

\section{The SST data}

The file `\textbf{SST.csv}' contains sea surface temperatures for Port
Nolloth and Muizenberg in °C. The data are from 1 January 2010 to 31
December 2011.

\begin{description}
\item[Question 12] Do the temperatures differ between the two places?
\item[Question 13] For each of the two sites, which month has the i) lowest and ii) highest temperature?
\item[Question 14] For each of the two sites, is the winter temperature colder than the summer temperature?
\item[Question 15] Same as Question 14, but use 95\% confidence intervals to approach this problem (and provide the supporting graphs).
\end{description}

Hint: The \textbf{lubridate} package (and others) offers convenient ways
to work with time series (i.e.~in this case coding a variable for
month).

\hfill \break

That's all, Folks!

\end{document}
